 In most practical applications of data mining the learning component
 is an integrated part of a far larger software environment. If the
 environment is written in Java, you can use WEKA to solve the
 learning problem without writing any machine learning code yourself.
 When invoking learning schemes from the graphical user interfaces or
 the command line, there is no need to know anything about programming
 in Java. In this section we show how to access these algorithms from
 your own Java code. From now on, we assume that you have at least
 some rudimentary knowledge of Java.

\section{A simple data mining application}

We present a simple data mining application for learning a model that
classifies text files into two categories, \textit{hit} and
\textit{miss}. The application works for arbitrary documents: we refer
to them as \textit{messages}. The implementation uses the
\textit{StringToWordVector} filter mentioned in
Section~\ref{subsubsection:string_conversion} to convert the
messages into attribute vectors. We assume that the program is called
every time a new file is to be processed. If the user provides a class
label for the file, the system uses it for training; if not, it
classifies it. The decision tree classifier \textit{J48} is used to do
the work.

\begin{figure}[!thp]
%\centering
\begin{mdframed}[innermargin=-1cm]
\begin{Verbatim}[fontsize=\footnotesize]
/**
 * Java program for classifying short text messages into two classes.
 */

import weka.classifiers.Classifier;
import weka.classifiers.trees.J48;
import weka.core.Attribute;
import weka.core.DenseInstance;
import weka.core.Instance;
import weka.core.Instances;
import weka.core.SerializationHelper;
import weka.core.Utils;
import weka.filters.Filter;
import weka.filters.unsupervised.attribute.StringToWordVector;

import java.io.FileNotFoundException;
import java.io.FileReader;
import java.io.Serializable;
import java.util.ArrayList;


public class MessageClassifier implements Serializable {

  /** for serialization. */
  private static final long serialVersionUID = -123455813150452885L;

  /** The training data gathered so far. */
  private Instances m_Data = null;

  /** The filter used to generate the word counts. */
  private StringToWordVector m_Filter = new StringToWordVector();

  /** The actual classifier. */
  private Classifier m_Classifier = new J48();

  /** Whether the model is up to date. */
  private boolean m_UpToDate;

  /**
   * Constructs empty training dataset.
   */
  public MessageClassifier() {
    String nameOfDataset = "MessageClassificationProblem";

    // Create vector of attributes.
    ArrayList<Attribute> attributes = new ArrayList<Attribute>(2);

    // Add attribute for holding messages.
    attributes.add(new Attribute("Message", (ArrayList<String>) null));

    // Add class attribute.
    ArrayList<String> classValues = new ArrayList<String>(2);
    classValues.add("miss");
    classValues.add("hit");
    attributes.add(new Attribute("Class", classValues));

    // Create dataset with initial capacity of 100, and set index of class.
    m_Data = new Instances(nameOfDataset, attributes, 100);
    m_Data.setClassIndex(m_Data.numAttributes() - 1);
  }
\end{Verbatim}
\end{mdframed}
\caption{\label{fig:message_classifier}The message classifier}
\end{figure}

\begin{figure}[!thp]
\ContinuedFloat
%\centering
\begin{mdframed}[innermargin=-1cm]
\begin{Verbatim}[fontsize=\footnotesize]
  /**
   * Updates model using the given training message.
   *
   * @param messagethe message content
   * @param classValuethe class label
   */
  public void updateData(String message, String classValue) {
    // Make message into instance.
    Instance instance = makeInstance(message, m_Data);

    // Set class value for instance.
    instance.setClassValue(classValue);

    // Add instance to training data.
    m_Data.add(instance);

    m_UpToDate = false;
  }

  /**
   * Classifies a given message.
   *
   * @param messagethe message content
   * @throws Exception if classification fails
   */
  public void classifyMessage(String message) throws Exception {
    // Check whether classifier has been built.
    if (m_Data.numInstances() == 0)
      throw new Exception("No classifier available.");

    // Check whether classifier and filter are up to date.
    if (!m_UpToDate) {
      // Initialize filter and tell it about the input format.
      m_Filter.setInputFormat(m_Data);

      // Generate word counts from the training data.
      Instances filteredData  = Filter.useFilter(m_Data, m_Filter);

      // Rebuild classifier.
      m_Classifier.buildClassifier(filteredData);

      m_UpToDate = true;
    }

    // Make separate little test set so that message
    // does not get added to string attribute in m_Data.
    Instances testset = m_Data.stringFreeStructure();

    // Make message into test instance.
    Instance instance = makeInstance(message, testset);

    // Filter instance.
    m_Filter.input(instance);
    Instance filteredInstance = m_Filter.output();

    // Get index of predicted class value.
    double predicted = m_Classifier.classifyInstance(filteredInstance);

    // Output class value.
    System.err.println("Message classified as : " +
           m_Data.classAttribute().value((int) predicted));
  }
\end{Verbatim}
\end{mdframed}
\caption{\label{fig:message_classifier}The message classifier cont.}
\end{figure}

\begin{figure}[!thp]
\ContinuedFloat
%\centering
\begin{mdframed}[innermargin=-1cm]
\begin{Verbatim}[fontsize=\scriptsize]
  /**
   * Method that converts a text message into an instance.
   *
   * @param textthe message content to convert
   * @param datathe header information
   * @returnthe generated Instance
   */
  private Instance makeInstance(String text, Instances data) {
    // Create instance of length two.
    Instance instance = new DenseInstance(2);

    // Set value for message attribute
    Attribute messageAtt = data.attribute("Message");
    instance.setValue(messageAtt, messageAtt.addStringValue(text));

    // Give instance access to attribute information from the dataset.
    instance.setDataset(data);

    return instance;
  }

  /**
   * Main method. The following parameters are recognized:
   *
   *   -m messagefile
   *      Points to the file containing the message to classify or use for
   *      updating the model.
   *   -c classlabel
   *      The class label of the message if model is to be updated. Omit for
   *      classification of a message.
   *   -t modelfile
   *      The file containing the model. If it doesn't exist, it will be
   *      created automatically.
   *
   * @param argsthe commandline options
   */
  public static void main(String[] args) {
    try {
      // Read message file into string.
      String messageName = Utils.getOption('m', args);
      if (messageName.length() == 0)
        throw new Exception("Must provide name of message file ('-m <file>').");
      FileReader m = new FileReader(messageName);
      StringBuffer message = new StringBuffer();
      int l;
      while ((l = m.read()) != -1)
      message.append((char) l);
      m.close();

      // Check if class value is given.
      String classValue = Utils.getOption('c', args);

      // If model file exists, read it, otherwise create new one.
      String modelName = Utils.getOption('t', args);
      if (modelName.length() == 0)
      throw new Exception("Must provide name of model file ('-t <file>').");
      MessageClassifier messageCl;
      try {
        messageCl = (MessageClassifier) SerializationHelper.read(modelName);
      }
      catch (FileNotFoundException e) {
        messageCl = new MessageClassifier();
      }

      // Check if there are any options left
      Utils.checkForRemainingOptions(args);

      // Process message.
      if (classValue.length() != 0)
        messageCl.updateData(message.toString(), classValue);
      else
        messageCl.classifyMessage(message.toString());

      // Save message classifier object only if it was updated.
      if (classValue.length() != 0)
      SerializationHelper.write(modelName, messageCl);
    }
    catch (Exception e) {
      e.printStackTrace();
    }
  }
}
\end{Verbatim}
\end{mdframed}
\caption{\label{fig:message_classifier}The message classifier cont.}
\end{figure}

Figure~\ref{fig:message_classifier} shows the source code for the
application program, implemented in a class called
\textit{MessageClassifier}. The command-line arguments that the
\textit{main()} method accepts are the name of a text file (given by
\textit{--m}), the name of a file holding an object of class
\textit{MessageClassifier} (\textit{--t}), and, optionally, the
classification of the message in the file (\textit{--c}). If the user
provides a classification, the message will be converted into an
example for training; if not, the \textit{MessageClassifier} object
will be used to classify it as hit or miss.

\section{main()}

The \textit{main()} method reads the message into a Java
\textit{StringBuffer} and checks whether the user has provided a
classification for it. Then it reads a \textit{MessageClassifier}
object from the file given by \textit{--t}, and creates a new object
of class \textit{MessageClassifier} if this file does not exist. In
either case the resulting object is called \textit{messageCl}. After
checking for illegal command-line options, the program calls the
method \textit{updateData()} message to update the training data
stored in \textit{messageCl} if a classification has been provided;
otherwise it calls \textit{classifyMessage()} to classify it. Finally,
the \textit{messageCl} object is saved back into the file, because it
may have changed. In the following, we first describe how a new
\textit{MessageClassifier} object is created by the constructor
\textit{MessageClassifier()} and then explain how the two methods
\textit{updateData()} and \textit{classifyMessage()} work.

\section{messageClassifier()}

Each time a new \textit{MessageClassifier} is created, objects for
holding the filter and classifier are generated automatically. The
only nontrivial part of the process is creating a dataset, which is
done by the constructor \textit{MessageClassifier()}. First the
dataset's name is stored as a string. Then an \textit{Attribute}
object is created for each attribute, one to hold the string
corresponding to a text message and the other for its class. These
objects are stored in a dynamic array of type \textit{ArrayList}.

Attributes are created by invoking one of the constructors in the
class \textit{Attribute}. This class has a constructor that takes one
parameter---the attribute's name---and creates a numeric
attribute. However, the constructor we use here takes two parameters:
the attribute's name and a reference to an \textit{ArrayList}. If this
reference is null, as in the first application of this constructor in
our program, WEKA creates an attribute of type
\textit{string}. Otherwise, a nominal attribute is created. This is
how we create a class attribute with two values \textit{hit} and
\textit{miss}: by passing the attribute's name (\textit{class}) and
its values---stored in an \textit{ArrayList}---to
\textit{Attribute()}.

To create a dataset from this attribute information,
\textit{MessageClassifier()} must create an object of the class
\textit{Instances} from the \textit{core} package. The constructor of
\textit{Instances} used by MessageClassifier() takes three arguments:
the dataset's name, a \textit{ArrayList} containing the attributes,
and an integer indicating the dataset's initial capacity. We set the
initial capacity to 100; it is expanded automatically if more
instances are added. After constructing the dataset,
\textit{MessageClassifier()} sets the index of the class attribute to
be the index of the last attribute.

\section{updateData()}

Now you know how to create an empty dataset, consider how the
\textit{MessageClassifier} object actually incorporates a new training
message. The method \textit{updateData()} does this. It first
converts the given message into a training instance by calling
\textit{makeInstance()}, which begins by creating an object of class
Instance that corresponds to an instance with two attributes. The
constructor of the Instance object sets all the instance's values to
be missing and its weight to 1. The next step in
\textit{makeInstance()} is to set the value of the string attribute
holding the text of the message. This is done by applying the
\textit{setValue()} method of the \textit{Instance} object, providing
it with the attribute whose value needs to be changed, and a second
parameter that corresponds to the new value's index in the definition
of the string attribute. This index is returned by the
\textit{addStringValue()} method, which adds the message text as a new
value to the string attribute and returns the position of this new
value in the definition of the string attribute.

Internally, an \textit{Instance} stores all attribute values as
double-precision floating-point numbers regardless of the type of the
corresponding attribute. In the case of nominal and string attributes
this is done by storing the index of the corresponding attribute value
in the definition of the attribute. For example, the first value of a
nominal attribute is represented by 0.0, the second by 1.0, and so
on. The same method is used for string attributes: \textit{addStringValue()}
returns the index corresponding to the value that is added to the
definition of the attribute.

Once the value for the string attribute has been set,
\textit{makeInstance()} gives the newly created instance access to the
data's attribute information by passing it a reference to the
dataset. In WEKA, an \textit{Instance} object does not store the type
of each attribute explicitly; instead, it stores a reference to a
dataset with the corresponding attribute information.

Returning to \textit{updateData()}, once the new instance has been
returned from \textit{makeInstance()} its class value is set and it is
added to the training data. We also initialize \textit{m\_UpToDate}, a
flag indicating that the training data has changed and the predictive
model is hence not up to date.

\section{classifyMessage()}

Now let's examine how \textit{MessageClassifier} processes a message
whose class label is unknown. The \textit{classifyMessage()} method
first checks whether a classifier has been built by determining
whether any training instances are available. It then checks whether
the classifier is up to date. If not (because the training data has
changed) it must be rebuilt. However, before doing so the data must be
converted into a format appropriate for learning using the
\textit{StringToWordVector} filter. First, we tell the filter the
format of the input data by passing it a reference to the input
dataset using \textit{setInputFormat()}. Every time this method is
called, the filter is initialized---that is, all its internal settings
are reset. In the next step, the data is transformed by
\textit{useFilter()}. This generic method from the \textit{Filter}
class applies a filter to a dataset. In this case, because
\textit{StringToWordVector} has just been initialized, it computes a
dictionary from the training dataset and then uses it to form a word
vector. After returning from \textit{useFilter()}, all the filter's
internal settings are fixed until it is initialized by another call of
\textit{setInputFormat()}. This makes it possible to filter a test
instance without updating the filter's internal settings (in this
case, the dictionary).

Once the data has been filtered, the program rebuilds the
classifier---in our case a \textit{J48} decision tree---by passing the
training data to its \textit{buildClassifier()} method. Then it sets
\textit{m\_UpToDate} to true. It is an important convention in WEKA
that the \textit{buildClassifier()} method completely initializes the
model's internal settings before generating a new classifier. Hence we
do not need to construct a new J48 object before we call
\textit{buildClassifier()}.

Having ensured that the model stored in \textit{m\_Classifier} is
current, we proceed to classify the message. Before
\textit{makeInstance()} is called to create an \textit{Instance}
object from it, a new \textit{Instances} object is created to hold the
new instance and passed as an argument to
\textit{makeInstance()}. This is done so that \textit{makeInstance()}
does not add the text of the message to the definition of the string
attribute in \textit{m\_Data}. Otherwise, the size of the
\textit{m\_Data} object would grow every time a new message was
classified, which is clearly not desirable---it should only grow when
training instances are added. Hence a temporary \textit{Instances}
object is created and discarded once the instance has been
processed. This object is obtained using the method
\textit{stringFreeStructure()}, which returns a copy of
\textit{m\_Data} with an empty string attribute. Only then is
\textit{makeInstance()} called to create the new instance.

The test instance must also be processed by the
\textit{StringToWordVector} filter before being classified. This is
easy: the \textit{input()} method enters the instance into the filter
object, and the transformed instance is obtained by calling
\textit{output()}. Then a prediction is produced by passing the
instance to the classifier's \textit{classifyInstance()} method. As
you can see, the prediction is coded as a double value. This allows
WEKA's evaluation module to treat models for categorical and numeric
prediction similarly. In the case of categorical prediction, as in
this example, the double variable holds the index of the predicted
class value. To output the string corresponding to this class value,
the program calls the \textit{value()} method of the dataset's class
attribute.

There is at least one way in which our implementation could be
improved. The classifier and the \textit{StringToWordVector} filter
could be combined using the \textit{FilteredClassifier} metalearner
described in Section~\ref{subsection:supervised_filters}. This
classifier would then be able to deal with string attributes directly,
without explicitly calling the filter to transform the data. We have not
done this here because we wanted to demonstrate how filters can be used
programmatically.

