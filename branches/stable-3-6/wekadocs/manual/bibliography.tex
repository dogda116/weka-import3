% Version: $Revision$

\begin{thebibliography}{999}
	% to make the bibliography appear in the TOC
	\addcontentsline{toc}{chapter}{Bibliography}

	% General
	\bibitem{witten}
		Witten, I.H. and Frank, E. (2005) \textit{Data Mining: Practical machine
		learning tools and techniques. 2nd edition}  Morgan Kaufmann, San
		Francisco.
	\bibitem{wekawiki}
		\textit{WekaWiki} -- \url{http://weka.wikispaces.com/}{}
%	\bibitem{wekadoc}
%		\textit{WekaDoc} -- \url{http://weka.sourceforge.net/wekadoc/}{}

	% command-line
	\bibitem{platt98}
		J. Platt (1998): Machines using Sequential Minimal Optimization. In B. Schoelkopf and C. Burges and A. Smola, editors, Advances in Kernel Methods - Support Vector Learning.

	% Explorer
	\bibitem{drummond}
		Drummond, C. and Holte, R. (2000) Explicitly representing expected cost: An alternative to ROC representation.
		\textit{Proceedings of the Sixth ACM SIGKDD International Conference on Knowledge Discovery and Data Mining.}
		Publishers, San Mateo, CA.
	%\bibitem{ensemble}
	%	\textit{Ensemble Selection} on \textit{WekaDoc} -- \\
	%	\small{\url{http://weka.sourceforge.net/wekadoc/index.php/en:Ensemble\_Selection}{}}
	\bibitem{mainextensions}
		\textit{Extensions for Weka's main GUI} on \textit{WekaWiki} -- \\
		\small{\url{http://weka.wikispaces.com/Extensions+for+Weka\%27s+main+GUI}{}}
	\bibitem{explorertabs}
		\textit{Adding tabs in the Explorer} on \textit{WekaWiki} -- \\
		\small{\url{http://weka.wikispaces.com/Adding+tabs+in+the+Explorer}{}}
	\bibitem{explorervisplugins}
		\textit{Explorer visualization plugins} on \textit{WekaWiki} -- \\
		\small{\url{http://weka.wikispaces.com/Explorer+visualization+plugins}{}}

	% Experimenter
	\bibitem{bengio}
		Bengio, Y. and Nadeau, C. (1999) \textit{Inference for the Generalization Error}.
	\bibitem{quinlan}
		Ross Quinlan (1993). \textit{C4.5: Programs for Machine Learning}, Morgan Kaufmann Publishers, San Mateo, CA.
	\bibitem{subversion}
		\textit{Subversion} -- \url{http://weka.wikispaces.com/Subversion}{}
	\bibitem{hsql}
		\textit{HSQLDB} -- \url{http://hsqldb.sourceforge.net/}{}
	\bibitem{mysql}
		\textit{MySQL} -- \url{http://www.mysql.com/}{}

	% KnowledgeFlow
	\bibitem{multipleroc}
		\textit{Plotting multiple ROC curves} on \textit{WekaWiki} -- \\
		\small{\url{http://weka.wikispaces.com/Plotting+multiple+ROC+curves}{}}

	% BayesNet
	\bibitem{bouck1995}
		R.R. Bouckaert. Bayesian Belief Networks: from Construction to Inference. 
		Ph.D. thesis, 
		University of Utrecht, 
		1995.
	\bibitem{Buntine1996}
		W.L. Buntine. A guide to the literature on learning probabilistic networks from data.
		IEEE Transactions on Knowledge and Data Engineering, 8:195--210, 
		1996. 
	\bibitem{ChengGreiner1999}
		J. Cheng, R. Greiner. 
		Comparing bayesian network classifiers. 
		%In Proceedings of the 15th Conference on Uncertainty in Artificial Intelligence (UAI'99), 
		Proceedings UAI,
		101--107,
		%. Morgan Kaufmann Publishers, August 
		1999.
	\bibitem{chow1968}
		C.K. Chow, C.N.Liu.
		Approximating discrete probability distributions with dependence trees.
		IEEE Trans. on Info. Theory, IT-14: 426--467, 1968.
	\bibitem{CooperHerskovits1992}
		G. Cooper, E. Herskovits. 
		A Bayesian method for the induction of probabilistic networks from data. 
		Machine Learning, 9: 309--347, 1992.
	\bibitem{cozman}
		Cozman.
		See {\sf \url{http://www-2.cs.cmu.edu/\~fgcozman/Research/InterchangeFormat/}{}}
		for details on XML BIF.
	\bibitem{friedman97} 
		N. Friedman, D. Geiger, M. Goldszmidt. 
		Bayesian Network Classifiers. 
		Machine Learning, 29: 131--163, 1997.
	\bibitem{heckerman95}	
		D. Heckerman, D. Geiger, D. M. Chickering. 
		Learning Bayesian networks: the combination of knowledge and statistical data. 
		Machine Learining, 20(3): 197--243, 1995.
	\bibitem{lauritzen}	
		S.L. Lauritzen and D.J. Spiegelhalter.
		Local Computations with Probabilities on graphical structures and their applications to expert systems (with discussion).
		Journal of the Royal Statistical Society B.
		1988, 50, 157-224
	\bibitem{Moore}
		Moore, A. and Lee, M.S. Cached Sufficient Statistics for Efficient Machine Learning with Large Datasets,
		JAIR, Volume 8, pages 67-91, 1998.
	\bibitem{verma}
		Verma, T. and Pearl, J.:
		An algorithm for deciding if a set of observed independencies has a causal explanation.
		Proc. of the Eighth Conference on Uncertainty in Artificial Intelligence,
		323-330, 1992.

	% Appendix

\end{thebibliography}
