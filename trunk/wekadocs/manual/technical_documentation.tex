% Version: $Revision$

\section{ANT}

What is ANT? This is how the ANT homepage (\verb=http://ant.apache.org/=) defines its tool:\\

\noindent \emph{Apache Ant is a Java-based build tool. In theory, it is kind of like Make, but without Make's wrinkles.}

\subsection{Basics}

\begin{itemize}
\item the ANT build file is based on XML
\item the usual name for the build file is:
\end{itemize}

\verb=  build.xml=

\begin{itemize}
\item invocation---the usual build file needs not be specified explicitly, if it's in the current directory; if not target is specified, the default one is used
\end{itemize}

\verb=  ant [-f <build-file>] [<target>]=

\begin{itemize}
\item displaying all the available targets of a build file
\end{itemize}

\verb=  ant [-f <build-file>] -projecthelp=

\subsection{Weka and ANT}

\begin{itemize}
\item a build file for Weka is available from subversion
\item some targets of interest
  \begin{itemize}
  \item \verb=clean=--Removes the build, dist and reports directories; also any class files in the source tree
  \item \verb=compile=--Compile weka and deposit class files in\\ \verb=${path_modifier}/build/classes=
  \item \verb=docs=--Make javadocs into \verb=${path_modifier}/doc=
  \item \verb=exejar=--Create an executable jar file in \verb=${path_modifier}/dist=
  \end{itemize}
\end{itemize}

\subsection{Links}
\begin{itemize}
\item ANT homepage: \verb=http://ant.apache.org/=
\item XML: \verb=http://www.w3.org/XML/=
\end{itemize}

% $http://weka.sourceforge.net/wekadoc/index.php/en:ANT_(3.5.6)$

\section{CLASSPATH}
The CLASSPATH environment variable tells Java where to look for
classes. Since Java does the search in a first-come-first-serve kind
of manner, you'll have to take care where and what to put in your
CLASSPATH. I, personally, never use the environment variable, since
I'm working often on a project in different versions in parallel. The
CLASSPATH would just mess up things, if you're not careful (or just
forget to remove an entry). ANT (\verb=http://ant.apache.org/=) offers
a nice way for building (and separating source code and class files)
Java projects. But still, if you're only working on totally separate
projects, it might be easiest for you to use the environment variable.

\subsection{Setting the CLASSPATH} 
In the following we add the \verb=mysql-connector-java-3.1.8-bin.jar=
to our \verb=CLASSPATH= variable (this works for any other jar archive) to make it possible to
access MySQL databases via JDBC.

\subsubsection*{Win32 (2k and XP)}
We assume that the \verbmysql-connector-java-3.1.8-bin.jar= archive is located in the following directory:\\

\verb=C:\Program Files\Weka-3-5=\\

\noindent In the \emph{Control Panel} click on \emph{System} (or right click on
\emph{My Computer} and select \emp{Properties}) and then go to the
\emph{Avanced} tab. There you'll find a button called \emph{Environment
Variables}, click it. Depending on, whether you're the only person
using this computer or it's a lab computer shared by many, you can
either create a new system-wide (you're the only user) environment
variable or a user dependent one (recommended for multi-user
machines). Enter the following name for the variable.\\

\verb=CLASSPATH=\\

\noindent and add this value\\

\verb=C:\Program Files\Weka-3-5\mysql-connector-java-3.1.8-bin.jar=\\

\noindent If you want to add additional jars, you'll have to separate them with the path separator, the semicolon ; (no spaces!). 

\subsubsection*{Unix/Linux}

We make the assumption that the mysql jar is located in the following directory:\\

\verb=/home/johndoe/jars/=\\

\noindent Open a shell and execute the following command, depending on the shell you're using:

\begin{itemize}
\item bash
\end{itemize}

\verb_export CLASSPATH=$CLASSPATH:/home/johndoe/jars/mysql-connector-java-3.1.8-bin.jar_

\begin{itemize}
\item c shell
\end{itemize}

\verb_setenv CLASSPATH $CLASSPATH:/home/johndoe/jars/mysql-connector-java-3.1.8-bin.jar_

\subsubsection*{Cygwin}

The process is like with Unix/Linux systems, but since the host system
is Win32 and therefore the Java installation also a Win32 application,
you'll have to use the semicolon ; as separator for several jars.

\subsection{RunWeka.bat}

With version 3.5.4, Weka is launched differently under Win32. The
simple batch file got replaced by a central launcher class
(\verb_= RunWeka.class_) in combination with an INI-file
 (\verb_= RunWeka.ini_). The \verb=RunWeka.bat= only calls this
 launcher class now with the appropriate parameters. With this
 launcher approach it is possible to define different launch
 scenarios, but with the advantage of having placeholders, e.g., for
 the max heap size, which enables one to change the memory for all
 setups easily.

The key of a command in the INI-file is prefixed with \verb=cmd_=, all
other keys are considered placeholders:\\

\verb^cmd_blah=java ...    ^ \emph{command \textbf{blah}}

\verb^bloerk= ...   ^ \emph{placeholder \textbf{bloerk}}\\

\noindent A placeholder is surrounded in a command with \#:\\

\verb^cmd_blah=java #bloerk#^\\

\noindent \textbf{Note:} The key \emph{wekajar} is determined by the
-w parameter with which the launcher class is called.\\

\noindent By default, the following commands are predefined:

\begin{itemize}
\item default\\ The default Weka start, without a terminal window.
\item console\\ For debugging purposes. Useful as Weka gets started from a terminal window.
\item explorer\\ The command that's executed if one double-clicks on an ARFF or XRFF file.
\end{itemize}

\noindent In order to change the \textbf{maximum heap size} for all those commands, one only has
to modify the \textbf{maxheap} placeholder.\\

\noindent For more information check out the comments in the INI-file.

\subsection{java -jar}

When you're using the Java interpreter with the \textbf{-jar} option,
be aware of the fact that it \textbf{overwrites} your \verb=CLASSPATH=
and not \textbf{augments it}. Out of convenience, people often only
use the \textbf{-jar} option to skip the declaration of the main class to
start. But as soon as you need more jars, e.g., for database access,
you need to use the \textbf{-classpath} option and specify the main class.\\

\noindent Here's once again how you start the Weka Main-GUI \textbf{with} your current \verb=CLASSPATH= variable (and 128MB for the JVM):

\begin{itemize}
\item Linux\\ \verb=java -Xmx128m -classpath $CLASSPATH:weka.jar weka.gui.Main=
\item Win32\\ \verb=java -Xmx128m -classpath "%CLASSPATH%;weka.jar" weka.gui.Main=
\end{itemize}

%$http://weka.sourceforge.net/wekadoc/index.php/en:CLASSPATH_(3.5.6)$

\section{Subversion}

\section{GenericObjectEditor}
$http://weka.sourceforge.net/wekadoc/index.php/en:GenericObjectEditor_(3.5.6)$

\section{Properties}
$file http://weka.sourceforge.net/wekadoc/index.php/en:Properties_File_(3.5.6)$

\section{XML}
$http://weka.sourceforge.net/wekadoc/index.php/en:XML_(3.5.6)$
