%
%    This program is free software; you can redistribute it and/or modify
%    it under the terms of the GNU General Public License as published by
%    the Free Software Foundation; either version 2 of the License, or
%    (at your option) any later version.
%
%    This program is distributed in the hope that it will be useful,
%    but WITHOUT ANY WARRANTY; without even the implied warranty of
%    MERCHANTABILITY or FITNESS FOR A PARTICULAR PURPOSE.  See the
%    GNU General Public License for more details.
%
%    You should have received a copy of the GNU General Public License
%    along with this program; if not, write to the Free Software
%    Foundation, Inc., 675 Mass Ave, Cambridge, MA 02139, USA.
%

% Version: $Revision$

The previous sections covered how to implement classifiers and filters. In the
following you will find some information on how to implement clusterers,
associators and attribute selection algorithms. The various algorithms are only
covered briefly, since other important components (capabilities, option
handling, revisions) have already been discussed in the other chapters.

%%%%%%%%%%%%%%
% Clusterers %
%%%%%%%%%%%%%%

\subsection{Clusterers}
\subsubsection*{Superclasses and interfaces}
All clusterers implement the interface \texttt{weka.clusterers.Clusterer},
but most algorithms will be most likely derived (directly or further up in the
class hierarchy) from the abstract superclass
\texttt{weka.clusterers.AbstractClusterer}.

\texttt{weka.clusterers.SingleClustererEnhancer} is used for meta-clusterers,
like the \texttt{FilteredClusterer} that filters the data on-the-fly for the
base-clusterer. \\

\noindent Here are some common interfaces that can be implemented:
\begin{tight_itemize}
  \item \texttt{weka.clusterers.DensityBasedClusterer} -- for clusterers that
can estimate the density for a given instance.
\texttt{AbstractDensityBasedClusterer} already implements this interface.
  \item \texttt{weka.clusterers.UpdateableClusterer} -- clusterers that can
generate their model incrementally implement this interface, like
\texttt{CobWeb}.
  \item \texttt{NumberOfClustersRequestable} -- is for clusterers that allow to
specify the number of clusters to generate, like \texttt{SimpleKMeans}.
  \item \texttt{weka.core.Randomizable} -- for clusterers that support
randomization in one way or another. \texttt{RandomizableClusterer},
\texttt{RandomizableDensityBasedClusterer} and
\texttt{RandomizableSingleClustererEnhancer} all implement this interface
already.
\end{tight_itemize}

\subsubsection*{Methods}
In the following a short description of methods that are common to all cluster
algorithms, see also the Javadoc for the \texttt{Clusterer} interface.

\heading{buildClusterer(Instances)}
Like the \texttt{buildClassifier(Instances)} method, this method completely
rebuilds the model. Subsequent calls of this method with the same dataset must
result in exactly the same model being built. This method also tests the
training data against the capabilities of this this clusterer:
\begin{verbatim}
  public void buildClusterer(Instances data) throws Exception {
    // test data against capabilities
    getCapabilities().testWithFail(data);
    // actual model generation
    ...
  }
\end{verbatim}

\heading{clusterInstance(Instance)}
returns the index of the cluster the provided \texttt{Instance} belongs to.

\heading{distributionForInstance(Instance)}
returns the cluster membership for this \texttt{Instance} object. The
membership is a double array containing the probabilities for each cluster.

\heading{numberOfClusters()}
returns the number of clusters that the model contains, after the model has
been generated with the \texttt{buildClusterer(Instances)} method.

\heading{getCapabilities()}
see section ``Capabilities'' on page \pageref{classifier_capabilities} for more
information.

\heading{toString()}
should output some information on the generated model. Even though this is not
required, it is rather useful for the user to get some feedback on the built
model.

\heading{main(String[])}
executes the clusterer from command-line. If your new algorithm is called
\texttt{FunkyClusterer}, then use the following code as your \texttt{main}
method:
\begin{verbatim}
  /**
   * Main method for executing this clusterer.
   *
   * @param args the options, use "-h" to display options
   */
  public static void main(String[] args) {
    AbstractClusterer.runClusterer(new FunkyClusterer(), args);
  }
\end{verbatim}

\subsubsection*{Testing}
For some basic tests from the command-line, you can use the following test
class:
\begin{verbatim}
  weka.clusterers.CheckClusterer -W classname [further options]
\end{verbatim}
For junit tests, you can subclass the
\texttt{weka.clusterers.AbstractClustererTest} class and add additional tests.

%%%%%%%%%%%%%%%%%%%%%%%
% Attribute selection %
%%%%%%%%%%%%%%%%%%%%%%%

\newpage
\subsection{Attribute selection}
Attribute selection consists basically of two different types of classes:
\begin{tight_itemize}
  \item \textit{evaluator} -- determines the merit of single attributes or
subsets of attributes
  \item \textit{search} algorithm -- the search heuristic
\end{tight_itemize}
Each of the them will be discussed separately in the following sections.

\subsubsection*{Evaluator}
The evaluator algorithm is responsible for determining merit of the current
attribute selection.

\heading{Superclasses and interfaces}
The ancestor for all evaluators is the
\texttt{weka.attributeSelection.ASEvaluation} class. \\

\noindent Here are some interfaces that are commonly implemented by evaluators:
\begin{tight_itemize}
  \item \texttt{AttributeEvaluator} -- evaluates only single attributes
  \item \texttt{SubsetEvaluator} -- evaluates subsets of attributes
  \item \texttt{AttributeTransformer} -- evaluators that transform the input
data
\end{tight_itemize}

\heading{Methods}
In the following a brief description of the main methods of an evaluator.

\heading{buildEvaluator(Instances)}
Generates the attribute evaluator. Subsequent calls of this method with the
same data (and the same search algorithm) must result in the same attributes
being selected. This method also checks the data against the capabilities:
\begin{verbatim}
  public void buildEvaluator (Instances data) throws Exception {
    // can evaluator handle data?
    getCapabilities().testWithFail(data);
    // actual initialization of evaluator
    ...
  }
\end{verbatim}

\heading{postProcess(int[])}
can be used for optional post-processing of the selected attributes, e.g., for
ranking purposes.

\newpage
\heading{main(String[])}
executes the evaluator from command-line. If your new algorithm is called
\texttt{FunkyEvaluator}, then use the following code as your \texttt{main}
method:
\begin{verbatim}
  /**
   * Main method for executing this evaluator.
   *
   * @param args the options, use "-h" to display options
   */
  public static void main(String[] args) {
    ASEvaluation.runEvaluator(new FunkyEvaluator(), args);
  }
\end{verbatim}

\subsubsection*{Search}
The search algorithm defines the heuristic of searching, e.g, exhaustive
search, greedy or genetic.

\heading{Superclasses and interfaces}
The ancestor for all search algorithms is the
\texttt{weka.attributeSelection.ASSearch} class. \\

\noindent Interfaces that can be implemented, if applicable, by a search
algorithm:
\begin{tight_itemize}
  \item \texttt{RankedOutputSearch} -- for search algorithms that produce
ranked lists of attributes
  \item \texttt{StartSetHandler} -- search algorithms that can make use of a
start set of attributes implement this interface
\end{tight_itemize}

\heading{Methods}
Search algorithms are rather basic classes in regards to methods that need to
be implemented. Only the following method needs to be implemented:

\heading{search(ASEvaluation,Instances)}
uses the provided evaluator to guide the search.

\subsubsection*{Testing}
For some basic tests from the command-line, you can use the following test
class:
\begin{verbatim}
  weka.attributeSelection.CheckAttributeSelection
    -eval classname -search classname [further options]
\end{verbatim}
For junit tests, you can subclass the
\texttt{weka.attributeSelection.AbstractEvaluatorTest} or
\texttt{weka.attributeSelection.AbstractSearchTest} class and add additional
tests.

%%%%%%%%%%%%%%%
% Associators %
%%%%%%%%%%%%%%%

\newpage
\subsection{Associators}
\subsubsection*{Superclasses and interfaces}
The interface \texttt{weka.associations.Associator} is common to all associator
algorithms. But most algorithms will be derived from
\texttt{AbstractAssociator}, an abstract class implementing this interface.
As with classifiers and clusterers, you can also implement a meta-associator,
derived from \texttt{SingleAssociatorEnhancer}. An example for this is the
\texttt{FilteredAssociator}, which filters the training data on-the-fly for the
base-associator.

The only other interface that is used by some other association algorithms, is
the \texttt{weka.clusterers.CARuleMiner} one. Associators that learn class
association rules implement this interface, like \texttt{Apriori}.

\subsubsection*{Methods}
The associators are very basic algorithms and only support building of the
model.

\heading{buildAssociations(Instances)}
Like the \texttt{buildClassifier(Instances)} method, this method completely
rebuilds the model. Subsequent calls of this method with the same dataset must
result in exactly the same model being built. This method also tests the
training data against the capabilities:
\begin{verbatim}
  public void buildAssociations(Instances data) throws Exception {
    // other necessary setups
    ...
    // test data against capabilities
    getCapabilities().testWithFail(data);
    // actual model generation
    ...
  }
\end{verbatim}

\heading{getCapabilities()}
see section ``Capabilities'' on page \pageref{classifier_capabilities} for more
information.

\heading{toString()}
should output some information on the generated model. Even though this is not
required, it is rather useful for the user to get some feedback on the built
model.

\newpage
\heading{main(String[])}
executes the associator from command-line. If your new algorithm is called
\texttt{FunkyAssociator}, then use the following code as your \texttt{main}
method:
\begin{verbatim}
  /**
   * Main method for executing this associator.
   *
   * @param args the options, use "-h" to display options
   */
  public static void main(String[] args) {
    AbstractAssociator.runAssociator(new FunkyAssociator(), args);
  }
\end{verbatim}

\subsubsection*{Testing}
For some basic tests from the command-line, you can use the following test
class:
\begin{verbatim}
  weka.associations.CheckAssociator -W classname [further options]
\end{verbatim}
For junit tests, you can subclass the
\texttt{weka.associations.AbstractAssociatorTest} class and add additional
tests.
