%
%    This program is free software; you can redistribute it and/or modify
%    it under the terms of the GNU General Public License as published by
%    the Free Software Foundation; either version 2 of the License, or
%    (at your option) any later version.
%
%    This program is distributed in the hope that it will be useful,
%    but WITHOUT ANY WARRANTY; without even the implied warranty of
%    MERCHANTABILITY or FITNESS FOR A PARTICULAR PURPOSE.  See the
%    GNU General Public License for more details.
%
%    You should have received a copy of the GNU General Public License
%    along with this program; if not, write to the Free Software
%    Foundation, Inc., 675 Mass Ave, Cambridge, MA 02139, USA.
%

% Version: $Revision$

Using the graphical tools, like the Explorer, or just the command-line is in
most cases sufficient for the normal user. But WEKA's clearly defined API
(``application programming interface'') makes it very easy to ``embed'' it in
another projects. This chapter covers the basics of how to achieve the
following common tasks from source code:
\begin{tight_itemize}
	\item Setting options
	\item Creating datasets in memory
	\item Loading and saving data
	\item Filtering
	\item Classifying
	\item Clustering
	\item Selecting attributes
	\item Visualization
	\item Serialization
\end{tight_itemize}
Even though most of the code examples are for the Linux platform, using forward
slashes in the paths and file names, they do work on the MS Windows platform as
well. To make the examples work under MS Windows, one only needs to adapt the
paths, changing the forward slashes to backslashes and adding a drive letter
where necessary. \\

\noindent \textbf{Note} \\
\noindent WEKA is released under the GNU General Public License version
2\footnote{\url{http://www.gnu.org/licenses/gpl-2.0.html}{}} (GPLv2), i.e., that
derived
code or code that uses WEKA needs to be released under the GPLv2 as well. If
one is just using WEKA for a personal project that does not get released
publicly then one is not affected. But as soon as one makes the project
publicly available (e.g., for download), then one needs to make the source code
available under the GLPv2 as well, alongside the binaries.

\newpage

%%%%%%%%%%%%%%%%%%%
% Option handling %
%%%%%%%%%%%%%%%%%%%
\section{Option handling}
\label{api_optionhandling}
Configuring an object, e.g., a classifier, can either be done using the
appropriate \texttt{get}/\texttt{set}-methods for the property that one wishes
to change, like the Explorer does. Or, if the class implements the
\texttt{weka.core.OptionHandler} interface, one can just use the object's
ability to parse command-line options via the \texttt{setOptions(String[])}
method (the counterpart of this method is \texttt{getOptions()}, which returns a
\texttt{String[]} array). The difference between the two approaches is, that the
\texttt{setOptions(String[])} method cannot be used to set the options
incrementally. Default values are used for all options that haven't been
explicitly specified in the options array.

The most basic approach is to assemble the \texttt{String} array by hand. The
following example creates an array with a single option (``\texttt{-R}'') that
takes an argument (``\texttt{1}'') and initializes the \texttt{Remove} filter
with this option:
\begin{verbatim}
  import weka.filters.unsupervised.attribute.Remove;
  ...
  String[] options = new String[2];
  options[0] = "-R";
  options[1] = "1";
  Remove rm = new Remove();
  rm.setOptions(options);
\end{verbatim}
Since the \texttt{setOptions(String[])} method expects a fully parsed and
correctly split up array (which is done by the console/command prompt), some
common pitfalls with this approach are:
\begin{tight_itemize}
	\item Combination of option and argument -- Using ``\texttt{-R 1}'' as an
element of the \texttt{String} array will fail, prompting WEKA to output an
error message stating that the option ``R 1'' is unknown.
	\item Trailing blanks -- Using ``\texttt{-R }'' will fail as well, since no
trailing blanks are removed and therefore option ``R '' will not be recognized.
\end{tight_itemize}
The easiest way to avoid these problems, is to provide a \texttt{String} array
that has been generated automatically from a single command-line string using
the \texttt{splitOptions(String)} method of the \texttt{weka.core.Utils} class.
Here is an example:
\begin{verbatim}
  import weka.core.Utils;
  ...
  String[] options = Utils.splitOptions("-R 1");
\end{verbatim}
As this method ignores whitespaces, using ``\texttt{  -R    1}'' or
``\texttt{-R 1 }'' will return the same result as ``\texttt{-R 1}''.

Complicated command-lines with lots of nested options, e.g., options for the
support-vector machine classifier \textit{SMO} (package
\texttt{weka.classifiers.functions}) including a kernel setup, are a bit tricky,
since Java requires one to escape double quotes and backslashes inside a
\texttt{String}. The Wiki\cite{wekawiki} article ``Use Weka in your Java code''
references the Java class \texttt{OptionsToCode}, which turns any command-line
into appropriate Java source code. This example class is also available from the
\textit{Weka Examples} collection\cite{wekaexamples}:
\texttt{weka.core.OptionsToCode}.

\newpage

Instead of using the \texttt{Remove} filter's \texttt{setOptions(String[])}
method, the following code snippet uses the actual \texttt{set}-method for this
property:
\begin{verbatim}
  import weka.filters.unsupervised.attribute.Remove;
  ...
  Remove rm = new Remove();
  rm.setAttributeIndices("1");
\end{verbatim}
In order to find out, which option belongs to which property, i.e.,
\texttt{get}/\texttt{set}-method, it is best to have a look at the
\texttt{setOptions(String[])} and \texttt{getOptions()} methods. In case these
methods use the member variables directly, one just has to look for the methods
making this particular member variable accessible to the outside.

Using the \texttt{set}-methods, one will most likely come across ones that
require a \texttt{weka.core.SelectedTag} as parameter. An example for this, is
the \texttt{setEvaluation} method of the meta-classifier \texttt{GridSearch}
(located in package \texttt{weka.classifiers.meta}). The \texttt{SelectedTag}
class is used in the GUI for displaying drop-down lists, enabling the user to
chose from a predefined list of values. \texttt{GridSearch} allows the user to
chose the statistical measure to base the evaluation on (accuracy, correlation
coefficient, etc.).

A \texttt{SelectedTag} gets constructed using the array of all possible
\texttt{weka.core.Tag} elements that can be chosen and the integer or string ID
of the \texttt{Tag}. For instance, \texttt{GridSearch}'s
\texttt{setOptions(String[])} method uses the supplied string ID to set the
evaluation type (e.g., ``\texttt{ACC}'' for accuracy), or, if the
evaluation option is missing, the default integer ID \texttt{EVALUATION\_ACC}.
In both cases, the array \texttt{TAGS\_EVALUATION} is used, which defines all
possible options:
\begin{verbatim}
   import weka.core.SelectedTag;
   ...
   String tmpStr = Utils.getOption('E', options);
   if (tmpStr.length() != 0)
     setEvaluation(new SelectedTag(tmpStr, TAGS_EVALUATION));
   else
     setEvaluation(new SelectedTag(EVALUATION_CC, TAGS_EVALUATION));
\end{verbatim}

\newpage

%%%%%%%%%%%%%%%%
% Loading data %
%%%%%%%%%%%%%%%%
\section{Loading data}
\label{api_loading_data}
Before any filter, classifier or clusterer can be applied, data needs to be
present. WEKA enables one to load data from files (in various file formats) and
also from databases. In the latter case, it is assumed in that the database
connection is set up and working. See chapter \ref{databases} for more details
on how to configure WEKA correctly and also more information on JDBC (Java
Database Connectivity) URLs.

Example classes, making use of the functionality covered in this section, can be
found in the \texttt{wekaexamples.core.converters} package of the \textit{Weka
Examples} collection\cite{wekaexamples}.

The following classes are used to store data in memory:
\begin{tight_itemize}
	\item \texttt{weka.core.Instances} -- holds a complete dataset. This data
structure is row-based; single rows can be accessed via the
\texttt{instance(int)} method using a 0-based index. Information about the
columns can be accessed via the \texttt{attribute(int)} method. This method
returns \texttt{weka.core.Attribute} objects (see below).
	\item \texttt{weka.core.Instance} -- encapsulates a single row. It is
basically a wrapper around an array of double primitives. Since this class
contains no information about the type of the columns, it always needs access
to a \texttt{weka.core.Instances} object (see methods \texttt{dataset} and
\texttt{setDataset}). The class \texttt{weka.core.SparseInstance} is used in
case of sparse data.
	\item \texttt{weka.core.Attribute} -- holds the type information about a
single column in the dataset. It stores the type of the attribute, as well as
the labels for \textit{nominal} attributes, the possible values for
\textit{string} attributes or the datasets for \textit{relational} attributes
(these are just \texttt{weka.core.Instances} objects again).
\end{tight_itemize}

\subsection{Loading data from files}
When loading data from files, one can either let WEKA choose the appropriate
loader (the available loaders can be found in the \texttt{weka.core.converters}
package) based on the file's extension or one can use the correct loader
directly. The latter case is necessary if the files do not have the correct
extension.

The \texttt{DataSource} class (inner class of the
\texttt{weka.core.converters.ConverterUtils} class) can be used to read data
from files that have the appropriate file extension. Here are some examples:
\begin{verbatim}
   import weka.core.converters.ConverterUtils.DataSource;
   import weka.core.Instances;
   ...
   Instances data1 = DataSource.read("/some/where/dataset.arff");
   Instances data2 = DataSource.read("/some/where/dataset.csv");
   Instances data3 = DataSource.read("/some/where/dataset.xrff");
\end{verbatim}
In case the file does have a different file extension than is normally
associated with the loader, one has to use a loader directly. The following
example loads a CSV (``comma-separated values'') file:
\begin{verbatim}
   import weka.core.converters.CSVLoader;
   import weka.core.Instances;
   import java.io.File;
   ...
   CSVLoader loader = new CSVLoader();
   loader.setSource(new File("/some/where/some.data"));
   Instances data = loader.getDataSet();
\end{verbatim}
\textbf{NB:} Not all file formats allow to store information about the class
attribute (e.g., ARFF stores no information about class attribute, but XRFF
does). If a class attribute is required further down the road, e.g., when using
a classifier, it can be set with the \texttt{setClassIndex(int)} method:
\begin{verbatim}
   // uses the first attribute as class attribute
   if (data.classIndex() == -1)
      data.setClassIndex(0);
   ...
   // uses the last attribute as class attribute
   if (data.classIndex() == -1)
      data.setClassIndex(data.numAttributes() - 1);
\end{verbatim}

\subsection{Loading data from databases}
For loading data from databases, one of the following two classes can be used:
\begin{tight_itemize}
	\item \texttt{weka.experiment.InstanceQuery}
	\item \texttt{weka.core.converters.DatabaseLoader}
\end{tight_itemize}
The differences between them are, that the \texttt{InstanceQuery} class allows
one to retrieve sparse data and the \texttt{DatabaseLoader} can retrieve the
data incrementally. \\

\noindent Here is an example of using the \texttt{InstanceQuery} class:
\begin{verbatim}
  import weka.core.Instances;
  import weka.experiment.InstanceQuery;
  ...
  InstanceQuery query = new InstanceQuery();
  query.setDatabaseURL("jdbc_url");
  query.setUsername("the_user");
  query.setPassword("the_password");
  query.setQuery("select * from whatsoever");
  // if your data is sparse, then you can say so, too:
  // query.setSparseData(true);
  Instances data = query.retrieveInstances();
\end{verbatim}
And an example using the \texttt{DatabaseLoader} class in ``batch retrieval'':
\begin{verbatim}
   import weka.core.Instances;
   import weka.core.converters.DatabaseLoader;
   ...
   DatabaseLoader loader = new DatabaseLoader();
   loader.setSource("jdbc_url", "the_user", "the_password");
   loader.setQuery("select * from whatsoever");
   Instances data = loader.getDataSet();
\end{verbatim}

\samepage
\noindent The \texttt{DatabaseLoader} is used in ``incremental mode'' as
follows:
\begin{verbatim}
   import weka.core.Instance;
   import weka.core.Instances;
   import weka.core.converters.DatabaseLoader;
   ...
   DatabaseLoader loader = new DatabaseLoader();
   loader.setSource("jdbc_url", "the_user", "the_password");
   loader.setQuery("select * from whatsoever");
   Instances structure = loader.getStructure();
   Instances data = new Instances(structure);
   Instance inst;
   while ((inst = loader.getNextInstance(structure)) != null)
      data.add(inst);
\end{verbatim}

\noindent \textbf{Notes:}
\begin{tight_itemize}
	\item Not all database systems allow incremental retrieval.
	\item Not all queries have a unique key to retrieve rows incrementally. In
that case, one can supply the necessary columns with the
\texttt{setKeys(String)} method (comma-separated list of columns).
	\item If the data cannot be retrieved in an incremental fashion, it is first
fully loaded into memory and then provided row-by-row (``pseudo-incremental'').
\end{tight_itemize}

\newpage

%%%%%%%%%%%%%%%%%%%%%%%%%%%%%%%
% Creating datasets in memory %
%%%%%%%%%%%%%%%%%%%%%%%%%%%%%%%
\section{Creating datasets in memory}
\label{api_creating_datasets}
Loading datasets from disk or database are not the only ways of obtaining data
in WEKA: datasets can be created in memory or \textit{on-the-fly}. Generating
a dataset memory structure (i.e., a \texttt{weka.core.Instances} object) is a
two-stage process:
\begin{tight_enumerate}
	\item Defining the format of the data by setting up the attributes.
	\item Adding the actual data, row by row.
\end{tight_enumerate}
The class \texttt{wekaexamples.core.CreateInstances} of the \textit{Weka
Examples} collection\cite{wekaexamples} generates an \texttt{Instances} object
containing all attribute types WEKA can handle at the moment.

\subsection{Defining the format}
There are currently five different types of attributes available in WEKA:
\begin{tight_itemize}
	\item \textbf{numeric} -- continuous variables
	\item \textbf{date} -- date variables
	\item \textbf{nominal} -- predefined labels
	\item \textbf{string} -- textual data
	\item \textbf{relational} -- contains other relations, e.g., the bags in
case of multi-instance data
\end{tight_itemize}
For all of the different attribute types, WEKA uses the same class,
\texttt{weka.core.Attribute}, but with different constructors. In the
following, these different constructors are explained.
\begin{tight_itemize}
	\item \textbf{numeric} -- The easiest attribute type to create, as it
requires only the name of the attribute:
	\begin{verbatim}
		Attribute numeric = new Attribute("name_of_attr");
	\end{verbatim}

	\item \texttt{date} -- Date attributes are handled internally as numeric
attributes, but in order to parse and present the date value correctly, the
format of the date needs to be specified. The \textit{date} and \textit{time
patterns} are explained in detail in the Javadoc of the
\texttt{java.text.SimpleDateFormat} class. In the following, an example of how
to create a date attribute using a date format of 4-digit year, 2-digit month
and a 2-digit day, separated by hyphens:
	\begin{verbatim}
		Attribute date = new Attribute("name_of_attr", "yyyy-MM-dd");
	\end{verbatim}

	\item \textbf{nominal} -- Since nominal attributes contain predefined
labels, one needs to supply these, stored in form of a
\texttt{java.util.ArrayList<String>} object:
	\begin{verbatim}
		ArrayList<String> labels = new ArrayList<String>();
		labels.addElement("label_a");
		labels.addElement("label_b");
		labels.addElement("label_c");
		labels.addElement("label_d");
		Attribute nominal = new Attribute("name_of_attr", labels);
	\end{verbatim}

	\item \textbf{string} -- In contrast to nominal attributes, this type does
not store a predefined list of labels. Normally used to store textual data,
i.e., content of documents for text categorization. The same constructor as for
the nominal attribute is used, but a \texttt{null} value is provided instead of
an instance of \texttt{java.util.ArrayList<String>}:
	\begin{verbatim}
		Attribute string = new Attribute("name_of_attr", (ArrayList<String>)
null);
	\end{verbatim}

	\newpage

	\item \texttt{relational} -- This attribute just takes another
\texttt{weka.core.Instances} object for defining the relational structure in
the constructor. The following code snippet generates a relational attribute
that contains a relation with two attributes, a numeric and a nominal attribute:
	\begin{verbatim}
		ArrayList<Attribute> atts = new ArrayList<Attribute>();
		atts.addElement(new Attribute("rel.num"));
		ArrayList<String> values = new ArrayList<String>();
		values.addElement("val_A");
		values.addElement("val_B");
		values.addElement("val_C");
		atts.addElement(new Attribute("rel.nom", values));
		Instances rel_struct = new Instances("rel", atts, 0);
		Attribute relational = new Attribute("name_of_attr", rel_struct);
	\end{verbatim}
\end{tight_itemize}
A \texttt{weka.core.Instances} object is then created by supplying a
\texttt{java.util.ArrayList<Attribute>} object containing all the attribute
objects. The following
example creates a dataset with two numeric attributes and a nominal class
attribute with two labels ``no'' and ``yes'':
\begin{verbatim}
   Attribute num1 = new Attribute("num1");
   Attribute num2 = new Attribute("num2");
   ArrayList<String> labels = new ArrayList<String>();
   labels.addElement("no");
   labels.addElement("yes");
   Attribute cls = new Attribute("class", labels);
   ArrayList<Attribute> attributes = new ArrayList<Attribute>();
   attributes.addElement(num1);
   attributes.addElement(num2);
   attributes.addElement(cls);
   Instances dataset = new Instances("Test-dataset", attributes, 0);
\end{verbatim}
The final argument in the \texttt{Instances} constructor above tells WEKA how
much memory to reserve for upcoming \texttt{weka.core.Instance} objects. If one
knows how many rows will be added to the dataset, then it should be specified
as it saves costly operations for expanding the internal storage. It doesn't
matter, if one aims to high with the amount of rows to be added, it is always
possible to \textit{trim} the dataset again, using the \texttt{compactify()}
method.

\subsection{Adding data}
After the structure of the dataset has been defined, one can add the actual
data to it, row by row. \texttt{weka.core.Instance} was turned into an
interface to provide greater flexibility. \texttt{weka.core.AbstractInstance}
implements this interface and provides basic functionality that is common to
\texttt{weka.core.DenseInstance} (formerly
\texttt{weka.core.Instance}) and \texttt{weka.core.SparseInstance} (which stores
only non-zero values). In the following examples, only the
\texttt{DenseInstance} class is used; the \texttt{SparseInstance} class is very
similar in handling.

There are basically two constructors of the \texttt{DenseInstance} class that
one can use for instantiating a data row:
\begin{tight_itemize}
	\item \texttt{DenseInstance(double weight, double[] attValues)} --
this constructor generates a \texttt{DenseInstance} object with the specified
weight and the given double values. WEKA's internal format is using doubles for
all attribute types. For nominal, string and relational attributes this is just
an index of the stored values.
	\item \texttt{DenseInstance(int numAttributes)} -- generates a new
\texttt{DenseInstance} object with weight 1.0 and all missing values.
\end{tight_itemize}
The second constructor may be easier to use, but setting values using the
methods of the \texttt{DenseInstance} is a bit costly, especially
if one is adding a lot of rows. Therefore, the following code examples cover the
first constructor. For simplicity, an \texttt{Instances} object
``\texttt{data}'' based on the code snippets for the different attribute
introduced used above is used, as it contains all possible attribute types.

For each instance, the first step is to create a new double array to hold the
attribute values. It is important not to reuse this array, but always create a
new one, since WEKA only references it and does not create a copy of it, when
instantiating the \texttt{DenseInstance} object. Reusing means changing the
previously generated \texttt{DenseInstance} object:
\begin{verbatim}
   double[] values = new double[data.numAttributes()];
\end{verbatim}
After that, the double array is filled with the actual values:
\begin{tight_itemize}
	\item \textbf{numeric} -- just sets the numeric value:
	\begin{verbatim}
	values[0] = 1.23;
	\end{verbatim}

	\item \textbf{date} -- turns the date string into a double value:
	\begin{verbatim}
	values[1] = data.attribute(1).parseDate("2001-11-09");
	\end{verbatim}

	\item \textbf{nominal} -- determines the index of the label:
	\begin{verbatim}
	values[2] = data.attribute(2).indexOf("label_b");
	\end{verbatim}

	\item \textbf{string} -- determines the index of the string, using the
\texttt{addStringValue} method (internally, a hashtable holds all the string
values):
	\begin{verbatim}
	values[3] = data.attribute(3).addStringValue("This is a string");
	\end{verbatim}

	\item \textbf{relational} -- first, a new \texttt{Instances} object based
on the attribute's relational definition has to be created, before the index
of it can be determined, using the \texttt{addRelation} method:
	\begin{verbatim}
	Instances dataRel = new Instances(data.attribute(4).relation(),0);
	valuesRel = new double[dataRel.numAttributes()];
	valuesRel[0] = 2.34;
	valuesRel[1] = dataRel.attribute(1).indexOf("val_C");
	dataRel.add(new DenseInstance(1.0, valuesRel));
	values[4] = data.attribute(4).addRelation(dataRel);
	\end{verbatim}
\end{tight_itemize}
Finally, an \texttt{Instance} object is generated with the initialized double
array and added to the dataset:
\begin{verbatim}
   Instance inst = new DenseInstance(1.0, values);
   data.add(inst);
\end{verbatim}


%%%%%%%%%%%%%%%%%%%%
% Randomizing data %
%%%%%%%%%%%%%%%%%%%%
\newpage
\section{Randomizing data}
\label{api_randomizing_data}
Since learning algorithms can be prone to the order the data arrives
in, randomizing (also called ``shuffling'') the data is a common approach to
alleviate this problem. Especially repeated randomizations, e.g., as during
cross-validation, help to generate more realistic statistics.

WEKA offers two possibilities for randomizing a dataset:
\begin{tight_itemize}
	\item Using the \texttt{randomize(Random)} method of the
\texttt{weka.core.Instances} object containing the data itself. This method
requires an instance of the \texttt{java.util.Random} class. How to correctly
instantiate such an object is explained below.
	\item Using the \texttt{Randomize} filter (package
\texttt{weka.filters.unsupervised.instance}). For more information on how to
use filters, see section \ref{api_filtering}.
\end{tight_itemize}
A very important aspect of Machine Learning experiments is, that experiments
have to be repeatable. Subsequent runs of the same experiment setup have to
yield the exact same results. It may seem weird, but randomization is still
possible in this scenario. Random number generates never return a completely
random sequence of numbers anyway, only a pseudo-random one. In order to
achieve repeatable pseudo-random sequences, \textit{seeded} generators are
used. Using the same \textit{seed value} will always result in the same
sequence then.

The default constructor of the \texttt{java.util.Random} random number generator
class should never be used, as such created objects will generate most likely
different sequences. The constructor \texttt{Random(long)}, using a specified 
seed value, is the recommended one to use.

In order to get a more dataset-dependent randomization of the data, the
\texttt{getRandomNumberGenerator(int)} method of the
\texttt{weka.core.Instances} class can be used. This method returns a
\texttt{java.util.Random} object that was seeded with the sum of the supplied
seed and the hashcode of the string representation of a randomly chosen
\texttt{weka.core.Instance} of the \texttt{Instances} object (using a random
number generator seeded with the seed supplied to this method).

\newpage

%%%%%%%%%%%%%
% Filtering %
%%%%%%%%%%%%%
\section{Filtering}
\label{api_filtering}
In WEKA, filters are used to preprocess the data. They can be found below
package \texttt{weka.filters}. Each filter falls into one of the following two
categories:
\begin{tight_itemize}
	\item \textit{supervised} -- The filter requires a class attribute to be
set.
	\item \textit{unsupervised} -- A class attribute is not required to be
present.
\end{tight_itemize}
And into one of the two sub-categories:
\begin{tight_itemize}
	\item \textit{attribute-based} -- Columns are processed, e.g., added or
removed.
	\item \textit{instance-based} -- Rows are processed, e.g., added or deleted.
\end{tight_itemize}
These categories should make it clear, what the difference between the two
\texttt{Discretize} filters in WEKA is. The \textit{supervised} one takes the
class attribute and its distribution over the dataset into account, in order to
determine the optimal number and size of bins, whereas the
\textit{unsupervised} one relies on a user-specified number of bins.

Apart from this classification, filters are either \textit{stream-} or
\textit{batch-based}. \textit{Stream} filters can process the data straight away
and make it immediately available for collection again. \textit{Batch} filters,
on the other hand, need a batch of data to setup their internal
data structures. The \texttt{Add} filter (this filter can be found in the
\texttt{weka.filters.unsupervised.attribute} package) is an example of a stream
filter. Adding a new attribute with only missing values does not require any
sophisticated setup. However, the \texttt{ReplaceMissingValues} filter (same
package as the \texttt{Add} filter) needs a batch of data in order to determine
the means and modes for each of the attributes. Otherwise, the filter will not
be able to replace the missing values with meaningful values. But as soon as a
batch filter has been initialized with the first batch of data, it can also
process data on a row-by-row basis, just like a stream filter.

\textit{Instance-based} filters are a bit special in the way they handle data.
As mentioned earlier, \textit{all} filters can process data on a row-by-row
basis after the first batch of data has been passed through. Of course, if a
filter adds or removes rows from a batch of data, this no longer works when
working in single-row processing mode. This makes sense, if one thinks of a
scenario involving the \texttt{FilteredClassifier} meta-classifier: after
the training phase (=\ first batch of data), the classifier will get
evaluated against a test set, one instance at a time. If the filter now
removes the only instance or adds instances, it can no longer be evaluated
correctly, as the evaluation expects to get only a single result back. This is
the reason why \textit{instance-based} filters only pass through any subsequent
batch of data without processing it. The \texttt{Resample} filters, for
instance, act like this.

One can find example classes for filtering in the \texttt{wekaexamples.filters}
package of the \textit{Weka Examples} collection\cite{wekaexamples}.

\newpage

The following example uses the \texttt{Remove} filter (the filter is located
in package \texttt{weka.filters.unsupervised.attribute}) to remove the
first attribute from a dataset. For setting the options, the
\texttt{setOptions(String[])} method is used.
\begin{verbatim}
   import weka.core.Instances;
   import weka.filters.Filter;
   import weka.filters.unsupervised.attribute.Remove;
   ...
   String[] options = new String[2];
   options[0] = "-R";                   // "range"
   options[1] = "1";                    // first attribute
   Remove remove = new Remove();        // new instance of filter
   remove.setOptions(options);          // set options
   remove.setInputFormat(data);         // inform filter about dataset
                                        // **AFTER** setting options
   Instances newData = Filter.useFilter(data, remove); // apply filter
\end{verbatim}
A common trap to fall into is setting options \textbf{after} the
\texttt{setInputFormat(Instances)} has been called. Since this method
is (normally) used to determine the output format of the data, \textbf{all} the
options have to be set \textbf{before} calling it. Otherwise, all options set
afterwards will be ignored.

\subsection{Batch filtering}
Batch filtering is necessary if two or more datasets need to be processed
according to the same filter initialization. If batch filtering is not used, 
for instance when generating a training and a test set using the
\texttt{StringToWordVector} filter (package
\texttt{weka.filters.unsupervised.attribute}), then these two filter runs are
 completely independent and will create two most likely incompatible datasets.
Running the \texttt{StringToWordVector} on two different datasets, this will
result in two different word dictionaries and therefore different attributes
being generated.

The following code example shows how to standardize, i.e., transforming all
numeric attributes to have zero mean and unit variance, a training and a test
set with the \texttt{Standardize} filter (package
\texttt{weka.filters.unsupervised.attribute}):
\begin{verbatim}
   Instances train = ...   // from somewhere
   Instances test = ...    // from somewhere
   Standardize filter = new Standardize();
   // initializing the filter once with training set
   filter.setInputFormat(train);
   // configures the Filter based on train instances and returns
   // filtered instances
   Instances newTrain = Filter.useFilter(train, filter);
   // create new test set
   Instances newTest = Filter.useFilter(test, filter);
\end{verbatim}

\newpage

\subsection{Filtering on-the-fly}
\label{api_filtering_onthefly}
Even though using the API gives one full control over the data and makes it
easier to juggle several datasets at the same time, filtering data
\textbf{on-the-fly} makes life even easier. This handy feature is available
through meta schemes in WEKA, like \texttt{FilteredClassifier} (package
\texttt{weka.classifiers.meta}), \texttt{FilteredClusterer} (package
\texttt{weka.clusterers}), \texttt{FilteredAssociator} (package
\texttt{weka.associations}) and
\texttt{FilteredAttributeEval}/\texttt{FilteredSubsetEval} (in
\texttt{weka.attributeSelection}). Instead of filtering the data
\textit{beforehand}, one just sets up a meta-scheme and lets the meta-scheme
do the filtering for one.

The following example uses the \texttt{FilteredClassifier} in conjunction with
the \texttt{Remove} filter to remove the first attribute (which happens to be
an ID attribute) from the dataset and \texttt{J48} (J48 is WEKA's
implementation of C4.5; package
\texttt{weka.classifiers.trees}) as base-classifier. First the classifier is
built with a training set and then evaluated with a separate test set. The
actual and predicted class values are printed in the console. For more
information on classification, see chapter \ref{api_classification}.
\begin{verbatim}
   import weka.classifiers.meta.FilteredClassifier;
   import weka.classifiers.trees.J48;
   import weka.core.Instances;
   import weka.filters.unsupervised.attribute.Remove;
   ...
   Instances train = ...         // from somewhere
   Instances test = ...          // from somewhere
   // filter
   Remove rm = new Remove();
   rm.setAttributeIndices("1");  // remove 1st attribute
   // classifier
   J48 j48 = new J48();
   j48.setUnpruned(true);        // using an unpruned J48
   // meta-classifier
   FilteredClassifier fc = new FilteredClassifier();
   fc.setFilter(rm);
   fc.setClassifier(j48);
   // train and output model
   fc.buildClassifier(train);
   System.out.println(fc);
   for (int i = 0; i < test.numInstances(); i++) {
     double pred = fc.classifyInstance(test.instance(i));
     double actual = test.instance(i).classValue();
     System.out.print("ID: "
       + test.instance(i).value(0));
     System.out.print(", actual: "
       + test.classAttribute().value((int) actual));
     System.out.println(", predicted: "
       + test.classAttribute().value((int) pred));
   }
\end{verbatim}

\newpage

%%%%%%%%%%%%%%%%%%
% Classification %
%%%%%%%%%%%%%%%%%%
\section{Classification}
\label{api_classification}
Classification and regression algorithms in WEKA are called ``classifiers'' and
are located below the \texttt{weka.classifiers} package. This section covers the
following topics:
\begin{tight_itemize}
	\item \textit{Building a classifier} -- batch and incremental learning.
	\item \textit{Evaluating a classifier} -- various evaluation techniques and
how to obtain the generated statistics.
	\item \textit{Classifying instances} -- obtaining classifications for
unknown data.
\end{tight_itemize}
The \textit{Weka Examples} collection\cite{wekaexamples} contains example
classes covering classification in the \texttt{wekaexamples.classifiers}
package.

\subsection{Building a classifier}
By design, all classifiers in WEKA are \textit{batch-trainable}, i.e., they get
trained on the whole dataset at once. This is fine, if the training data fits
into memory. But there are also algorithms available that can update their
internal model \textit{on-the-go}. These classifiers are called
\textit{incremental}. The following two sections cover the batch and the
incremental classifiers.

\subsubsection*{Batch classifiers}
A batch classifier is really simple to build:
\begin{tight_itemize}
	\item \textit{set options} -- either using the \texttt{setOptions(String[])}
method or the actual set-methods.
	\item \textit{train it} -- calling the \texttt{buildClassifier(Instances)}
method with the training set. By definition, the
\texttt{buildClassifier(Instances)} method resets the internal model completely,
in order to ensure that subsequent calls of this method with the same data
result in the same model (``repeatable experiments'').
\end{tight_itemize}
The following code snippet builds an unpruned J48 on a dataset:
\begin{verbatim}
   import weka.core.Instances;
   import weka.classifiers.trees.J48;
   ...
   Instances data = ...              // from somewhere
   String[] options = new String[1];
   options[0] = "-U";                // unpruned tree
   J48 tree = new J48();             // new instance of tree
   tree.setOptions(options);         // set the options
   tree.buildClassifier(data);       // build classifier
\end{verbatim}

\subsubsection*{Incremental classifiers}
All incremental classifiers in WEKA implement the interface
\texttt{UpdateableClassifier} (located in package \texttt{weka.classifiers}).
Bringing up the Javadoc for this particular interface tells one what classifiers
implement this interface. These classifiers can be used to process large amounts
of data with a small memory-footprint, as the training data does not have to fit
in memory. ARFF files, for instance, can be read incrementally (see chapter
\ref{api_loading_data}).

Training an incremental classifier happens in \textit{two} stages:
\begin{tight_enumerate}
	\item \textit{initialize} the model by calling the
\texttt{buildClassifier(Instances)} method. One can either use a
\texttt{weka.core.Instances} object with no actual data or one with an initial
set of data.
	\item \textit{update} the model row-by-row, by calling the
\texttt{updateClassifier(Instance)} method.
\end{tight_enumerate}
The following example shows how to load an ARFF file incrementally using the
\texttt{ArffLoader} class and train the \texttt{NaiveBayesUpdateable} classifier
with one row at a time:
\begin{verbatim}
   import weka.core.converters.ArffLoader;
   import weka.classifiers.bayes.NaiveBayesUpdateable;
   import java.io.File;
   ...
   // load data
   ArffLoader loader = new ArffLoader();
   loader.setFile(new File("/some/where/data.arff"));
   Instances structure = loader.getStructure();
   structure.setClassIndex(structure.numAttributes() - 1);

   // train NaiveBayes
   NaiveBayesUpdateable nb = new NaiveBayesUpdateable();
   nb.buildClassifier(structure);
   Instance current;
   while ((current = loader.getNextInstance(structure)) != null)
     nb.updateClassifier(current);
\end{verbatim}

\newpage

\subsection{Evaluating a classifier}
Building a classifier is only one part of the equation, evaluating how
\textit{well} it performs is another important part. WEKA supports two types of
evaluation:
\begin{tight_itemize}
	\item \textit{Cross-validation} -- If one only has a single dataset and
wants to get a reasonable realistic evaluation. Setting the number of folds equal
to the number of rows in the dataset will give one leave-one-out
cross-validation (LOOCV).
	\item \textit{Dedicated test set} -- The test set is solely used to evaluate
the built classifier. It is important to have a test set that incorporates the
same (or similar) concepts as the training set, otherwise one will always end
up with poor performance.
\end{tight_itemize}
The evaluation step, including collection of statistics, is performed by the
\texttt{Evaluation} class (package \texttt{weka.classifiers}).

\subsubsection*{Cross-validation}
The \texttt{crossValidateModel} method of the \texttt{Evaluation} class is used
to perform cross-validation with an \textbf{untrained} classifier and a single
dataset. Supplying an untrained classifier ensures that no information leaks
into the actual evaluation. Even though it is an implementation requirement,
that the \texttt{buildClassifier} method resets the classifier, it cannot be
guaranteed that this is indeed the case (``leaky'' implementation). Using an
untrained classifier avoids unwanted side-effects, as for each train/test set
pair, a copy of the originally supplied classifier is used.

Before cross-validation is performed, the data gets randomized using the
supplied random number generator (\texttt{java.util.Random}). It is
recommended that this number generator is ``seeded'' with a specified seed
value. Otherwise, subsequent runs of cross-validation on the same dataset will
not yield the same results, due to different randomization of the data (see
section \ref{api_randomizing_data} for more information on randomization).

The code snippet below performs 10-fold cross-validation with a J48 decision
tree algorithm on a dataset \texttt{newData}, with random number generator that
is seeded with ``1''. The summary of the collected statistics is output to
\texttt{stdout}.

\newpage

\begin{verbatim}
   import weka.classifiers.Evaluation;
   import weka.classifiers.trees.J48;
   import weka.core.Instances;
   import java.util.Random;
   ...
   Instances newData = ... // from somewhere
   Evaluation eval = new Evaluation(newData);
   J48 tree = new J48();
   eval.crossValidateModel(tree, newData, 10, new Random(1));
   System.out.println(eval.toSummaryString("\nResults\n\n", false));
\end{verbatim}
The \texttt{Evaluation} object in this example is initialized with the dataset
used in the evaluation process. This is done in order to inform the evaluation
about the type of data that is being evaluated, ensuring that all internal data
structures are setup correctly.

\subsubsection*{Train/test set}
Using a dedicated test set to evaluate a classifier is just as easy as
cross-validation. But instead of providing an untrained classifier, a trained
classifier has to be provided now. Once again, the
\texttt{weka.classifiers.Evaluation} class is used to perform the evaluation,
this time using the \texttt{evaluateModel} method.

The code snippet below trains a J48 with default options on a training set and
evaluates it on a test set before outputting the summary of the
collected statistics:
\begin{verbatim}
   import weka.core.Instances;
   import weka.classifiers.Evaluation;
   import weka.classifiers.trees.J48;
   ...
   Instances train = ...   // from somewhere
   Instances test = ...    // from somewhere
   // train classifier
   Classifier cls = new J48();
   cls.buildClassifier(train);
   // evaluate classifier and print some statistics
   Evaluation eval = new Evaluation(train);
   eval.evaluateModel(cls, test);
   System.out.println(eval.toSummaryString("\nResults\n\n", false));
\end{verbatim}

\newpage

\subsubsection*{Statistics}
In the previous sections, the \texttt{toSummaryString} of the
\texttt{Evaluation} class was already used in the code examples. But there are
other summary methods for nominal class attributes available as well:
\begin{tight_itemize}
	\item \texttt{toMatrixString} -- outputs the confusion matrix.
	\item \texttt{toClassDetailsString} -- outputs TP/FP rates, precision,
recall, F-measure, AUC (per class).
	\item \texttt{toCumulativeMarginDistributionString} -- outputs the
cumulative margins distribution.
\end{tight_itemize}
If one does not want to use these summary methods, it is possible to access the
individual statistical measures directly. Below, a few common measures
are listed:
\begin{tight_itemize}
	\item nominal class attribute
	\begin{tight_itemize}
		\item \texttt{correct()} -- The number of correctly classified
instances. The incorrectly classified ones are available through
\texttt{incorrect()}.
		\item \texttt{pctCorrect()} -- The percentage of correctly classified
instances (accuracy). \texttt{pctIncorrect()} returns the number of
misclassified ones.
		\item \texttt{areaUnderROC(int)} -- The AUC for the specified class
label index (0-based index).
	\end{tight_itemize}

	\item numeric class attribute
	\begin{tight_itemize}
		\item \texttt{correlationCoefficient()} -- The correlation coefficient.
	\end{tight_itemize}

	\item general
	\begin{tight_itemize}
		\item \texttt{meanAbsoluteError()} -- The mean absolute error.
		\item \texttt{rootMeanSquaredError()} -- The root mean squared error.
		\item \texttt{numInstances()} -- The number of instances with a class
value.
		\item \texttt{unclassified()} - The number of unclassified instances.
		\item \texttt{pctUnclassified()} - The percentage of unclassified
instances.
	\end{tight_itemize}
\end{tight_itemize}
For a complete overview, see the Javadoc page of the \texttt{Evaluation} class.
By looking up the source code of the summary methods mentioned above, one can
easily determine what methods are used for which particular output.

\newpage

\subsection{Classifying instances}
After a classifier setup has been evaluated and proven to be useful, a built
classifier can be used to make predictions and label previously unlabeled data.
Section \ref{api_filtering_onthefly} already provided a glimpse of how to use a
classifier's \texttt{classifyInstance} method. This section here elaborates a
bit more on this.

The following example uses a trained classifier \texttt{tree} to label all the
instances in an unlabeled dataset that gets loaded from disk. After all the
instances have been labeled, the newly labeled dataset gets written back to
disk to a new file.
\begin{verbatim}
   // load unlabeled data and set class attribute
   Instances unlabeled = DataSource.read("/some/where/unlabeled.arff"); 
   unlabeled.setClassIndex(unlabeled.numAttributes() - 1);
   // create copy
   Instances labeled = new Instances(unlabeled);
   // label instances
   for (int i = 0; i < unlabeled.numInstances(); i++) {
     double clsLabel = tree.classifyInstance(unlabeled.instance(i));
     labeled.instance(i).setClassValue(clsLabel);
   }
   // save newly labeled data
   DataSink.write("/some/where/labeled.arff", labeled);
\end{verbatim}
The above example works for classification and regression problems alike, as
long as the classifier can handle numeric classes, of course. Why is that? The
\texttt{classifyInstance(Instance)} method returns for numeric classes the
regression value and for nominal classes the 0-based index in the list of
available class labels.

If one is interested in the class distribution instead, then one can use the
\texttt{distributionForInstance(Instance)} method (this array sums up to 1). Of
course, using this method makes only sense for classification problems. The code
snippet below outputs the class distribution, the actual and predicted label
side-by-side in the console:
\begin{verbatim}
   // load data
   Instances train = DataSource.read(args[0]);
   train.setClassIndex(train.numAttributes() - 1);
   Instances test = DataSource.read(args[1]);
   test.setClassIndex(test.numAttributes() - 1);
   // train classifier
   J48 cls = new J48();
   cls.buildClassifier(train);
   // output predictions
   System.out.println("# - actual - predicted - distribution");
   for (int i = 0; i < test.numInstances(); i++) {
     double pred = cls.classifyInstance(test.instance(i));
     double[] dist = cls.distributionForInstance(test.instance(i));
     System.out.print((i+1) + " - ");
     System.out.print(test.instance(i).toString(test.classIndex()) + " - ");
     System.out.print(test.classAttribute().value((int) pred) + " - ");
     System.out.println(Utils.arrayToString(dist));
   }
\end{verbatim}

\newpage

%%%%%%%%%%%%%%
% Clustering %
%%%%%%%%%%%%%%
\section{Clustering}
\label{api_clustering}
Clustering is an unsupervised Machine Learning technique of finding patterns in
the data, i.e., these algorithms work without class attributes. Classifiers, on
the other hand, are supervised and need a class attribute. This section,
similar to the one about classifiers, covers the following topics:
\begin{tight_itemize}
	\item \textit{Building a clusterer} -- batch and incremental learning.
	\item \textit{Evaluating a clusterer} -- how to evaluate a built clusterer.
	\item \textit{Clustering instances} -- determining what clusters unknown
instances belong to.
\end{tight_itemize}
Fully functional example classes are located in the
\texttt{wekaexamples.clusterers} package of the \textit{Weka Examples}
collection\cite{wekaexamples}.

\subsection{Building a clusterer}
Clusterers, just like classifiers, are by design batch-trainable as well. They
all can be built on data that is completely stored in memory. But a small subset
of the cluster algorithms can also update the internal representation
incrementally. The following two sections cover both types of clusterers.

\subsubsection*{Batch clusterers}
Building a batch clusterer, just like a classifier, happens in two stages:
\begin{tight_itemize}
	\item \textit{set options} -- either calling the
\texttt{setOptions(String[])} method or the appropriate \texttt{set}-methods of
the properties.
	\item \textit{build the model} with training data -- calling the
\texttt{buildClusterer(Instances)} method. By definition, subsequent calls of
this method must result in the same model (``repeatable experiments''). In other
words, calling this method must completely reset the model.
\end{tight_itemize}
Below is an example of building the \texttt{EM} clusterer with a maximum of 100
iterations. The options are set using the \texttt{setOptions(String[])} method:
\begin{verbatim}
   import weka.clusterers.EM;
   import weka.core.Instances;
   ...
   Instances data = ... // from somewhere
   String[] options = new String[2];
   options[0] = "-I";                 // max. iterations
   options[1] = "100";
   EM clusterer = new EM();   // new instance of clusterer
   clusterer.setOptions(options);     // set the options
   clusterer.buildClusterer(data);    // build the clusterer
\end{verbatim}

\subsubsection*{Incremental clusterers}
Incremental clusterers in WEKA implement the interface
\texttt{UpdateableClusterer} (package \texttt{weka.clusterers}). Training an
incremental clusterer happens in three stages, similar to incremental
classifiers:
\begin{tight_enumerate}
	\item \textit{initialize} the model by calling the
\texttt{buildClusterer(Instances)} method. Once again, one can either use an
empty \texttt{weka.core.Instances} object or one with an initial set of data.
	\item \textit{update} the model row-by-row by calling the the
\texttt{updateClusterer(Instance)} method.
	\item \textit{finish} the training by calling
\texttt{updateFinished()} method. In case cluster algorithms need to perform
computational expensive post-processing or clean up operations.
\end{tight_enumerate}

\newpage

An \texttt{ArffLoader} is used in the following example to build the
\texttt{Cobweb} clusterer incrementally:
\begin{verbatim}
   import weka.clusterers.Cobweb;
   import weka.core.Instance;
   import weka.core.Instances;
   import weka.core.converters.ArffLoader;
   ...
   // load data
   ArffLoader loader = new ArffLoader();
   loader.setFile(new File("/some/where/data.arff"));
   Instances structure = loader.getStructure();
   // train Cobweb
   Cobweb cw = new Cobweb();
   cw.buildClusterer(structure);
   Instance current;
   while ((current = loader.getNextInstance(structure)) != null)
     cw.updateClusterer(current);
   cw.updateFinished();
\end{verbatim}

\newpage

\subsection{Evaluating a clusterer}
Evaluation of clusterers is not as comprehensive as the evaluation of
classifiers. Since clustering is unsupervised, it is also a lot harder
determining how \textit{good} a model is. The class used for evaluating cluster
algorithms, is \texttt{ClusterEvaluation} (package \texttt{weka.clusterers}).

In order to generate the same output as the Explorer or the command-line, one
can use the \texttt{evaluateClusterer} method, as shown below:
\begin{verbatim}
  import weka.clusterers.EM;
  import weka.clusterers.ClusterEvaluation;
  ...
  String[] options = new String[2];
  options[0] = "-t";
  options[1] = "/some/where/somefile.arff";
  System.out.println(ClusterEvaluation.evaluateClusterer(new EM(), options));
\end{verbatim}
Or, if the dataset is already present in memory, one can use the following
approach:
\begin{verbatim}
   import weka.clusterers.ClusterEvaluation;
   import weka.clusterers.EM;
   import weka.core.Instances;
   ...
   Instances data = ... // from somewhere
   EM cl = new EM();
   cl.buildClusterer(data);
   ClusterEvaluation eval = new ClusterEvaluation();
   eval.setClusterer(cl);
   eval.evaluateClusterer(new Instances(data));
   System.out.println(eval.clusterResultsToString());
\end{verbatim}
Density based clusterers, i.e., algorithms that implement the interface named
\texttt{DensityBasedClusterer} (package \texttt{weka.clusterers}) can
be cross-validated and the log-likelyhood obtained. Using the
\texttt{MakeDensityBasedClusterer} meta-clusterer, any non-density based
clusterer can be turned into such. Here is an example of cross-validating a
density based clusterer and obtaining the log-likelyhood:
\begin{verbatim}
   import weka.clusterers.ClusterEvaluation;
   import weka.clusterers.DensityBasedClusterer;
   import weka.core.Instances;
   import java.util.Random;
   ...
   Instances data = ...                         // from somewhere
   DensityBasedClusterer clusterer = new ...    // the clusterer to evaluate
   double logLikelyhood =
   ClusterEvaluation.crossValidateModel(        // cross-validate
       clusterer, data, 10,                     // with 10 folds
       new Random(1));                          // and random number generator
                                                // with seed 1
\end{verbatim}

\newpage

\subsubsection*{Classes to clusters}
Datasets for supervised algorithms, like classifiers, can be used to
evaluate a clusterer as well. This evaluation is called
\textit{classes-to-clusters}, as the clusters are mapped back onto the classes.

This type of evaluation is performed as follows:
\begin{tight_enumerate}
	\item \textit{create a copy} of the dataset containing the class attribute
and remove the class attribute, using the \texttt{Remove} filter (this
filter is located in package \texttt{weka.filters.unsupervised.attribute}).
	\item \textit{build the clusterer} with this new data.
	\item \textit{evaluate the clusterer} now with the \textbf{original} data.
\end{tight_enumerate}
And here are the steps translated into code, using \texttt{EM} as the
clusterer being evaluated:
\begin{tight_enumerate}
	\item create a copy of data without class attribute
	\begin{verbatim}
    Instances data = ... // from somewhere
    Remove filter = new Remove();
    filter.setAttributeIndices("" + (data.classIndex() + 1));
    filter.setInputFormat(data);
    Instances dataClusterer = Filter.useFilter(data, filter);
	\end{verbatim}

	\item build the clusterer
	\begin{verbatim}
    EM clusterer = new EM();
    // set further options for EM, if necessary...
    clusterer.buildClusterer(dataClusterer);
	\end{verbatim}

	\item evaluate the clusterer
	\begin{verbatim}
    ClusterEvaluation eval = new ClusterEvaluation();
    eval.setClusterer(clusterer);
    eval.evaluateClusterer(data);
    // print results
    System.out.println(eval.clusterResultsToString());
	\end{verbatim}
\end{tight_enumerate}

\newpage

\subsection{Clustering instances}
Clustering of instances is very similar to classifying unknown instances when
using classifiers. The following methods are involved:
\begin{tight_itemize}
	\item \texttt{clusterInstance(Instance)} -- determines the cluster the
\texttt{Instance} would belong to.
	\item \texttt{distributionForInstance(Instance)} -- predicts the cluster
membership for this \texttt{Instance}. The sum of this array adds up to 1.
\end{tight_itemize}
The code fragment outlined below trains an \texttt{EM} clusterer on one
dataset and outputs for a second dataset the predicted clusters and cluster
memberships of the individual instances:
\begin{verbatim}
   import weka.clusterers.EM;
   import weka.core.Instances;
   ...
   Instances dataset1 = ... // from somewhere
   Instances dataset2 = ... // from somewhere
   // build clusterer
   EM clusterer = new EM();
   clusterer.buildClusterer(dataset1);
   // output predictions
   System.out.println("# - cluster - distribution");
   for (int i = 0; i < dataset2.numInstances(); i++) {
     int cluster = clusterer.clusterInstance(dataset2.instance(i));
     double[] dist = clusterer.distributionForInstance(dataset2.instance(i));
     System.out.print((i+1));
     System.out.print(" - ");
     System.out.print(cluster);
     System.out.print(" - ");
     System.out.print(Utils.arrayToString(dist));
     System.out.println();
   }
\end{verbatim}

\newpage

%%%%%%%%%%%%%%%%%%%%%%%%
% Selecting attributes %
%%%%%%%%%%%%%%%%%%%%%%%%
\section{Selecting attributes}
\label{api_selecting_attributes}
Preparing one's data properly is a very important step for getting the best
results. Reducing the number of attributes can not only help speeding up
runtime with algorithms (some algorithms' runtime are quadratic in regards to
number of attributes), but also help avoid ``burying'' the algorithm in a
mass of attributes, when only a few are essential for building a good model. \\

\noindent There are three different types of evaluators in WEKA at the moment:
\begin{tight_itemize}
	\item \textit{single attribute evaluators} -- perform
evaluations on single attributes. These classes implement the
\texttt{weka.attributeSelection.AttributeEvaluator} interface. The
\texttt{Ranker} search algorithm is usually used in conjunction with these
algorithms.
	\item \textit{attribute subset evaluators} -- work on subsets of all the
attributes in the dataset. The \texttt{weka.attributeSelection.SubsetEvaluator}
interface is implemented by these evaluators.
	\item \textit{attribute set evaluators} -- evaluate sets of attributes. Not
to be confused with the \textit{subset evaluators}, as these classes are derived
from the \texttt{weka.attributeSelection.AttributeSetEvaluator} superclass.
\end{tight_itemize}
Most of the attribute selection schemes currently implemented are supervised,
i.e., they require a dataset with a class attribute. Unsupervised evaluation
algorithms
are derived from one of the following superclasses:
\begin{tight_itemize}
	\item \texttt{weka.attributeSelection.UnsupervisedAttributeEvaluator} \\
		e.g., \texttt{LatentSemanticAnalysis}, \texttt{PrincipalComponents}
	\item \texttt{weka.attributeSelection.UnsupervisedSubsetEvaluator} \\
		none at the moment
\end{tight_itemize}
Attribute selection offers \textit{filtering on-the-fly}, like classifiers
and clusterers, as well:
\begin{tight_itemize}
	\item \texttt{weka.attributeSelection.FilteredAttributeEval} -- filter for
evaluators that evaluate attributes individually.
	\item \texttt{weka.attributeSelection.FilteredSubsetEval} -- for
filtering evaluators that evaluate subsets of attributes.
\end{tight_itemize}
So much about the differences among the various attribute selection
algorithms and back to how to actually perform attribute selection. WEKA offers
three different approaches:
\begin{tight_itemize}
	\item \textit{Using a meta-classifier} -- for performing attribute selection
on-the-fly (similar to FilteredClassifier's filtering on-the-fly).
	\item \textit{Using a filter} - for preprocessing the data.
	\item \textit{Low-level API usage} - instead of using the meta-schemes
(classifier or filter), one can use the attribute selection API directly as
well.
\end{tight_itemize}
The following sections cover each of the topics, accompanied with a code
example. For clarity, the same evaluator and search algorithm is used in all
of these examples.

Feel free to check out the example classes of the \textit{Weka Examples}
collection\cite{wekaexamples}, located in the
\texttt{wekaexamples.attributeSelection} package.

\newpage

\subsection{Using the meta-classifier}
The meta-classifier \texttt{AttributeSelectedClassifier} (this classifier
is located in package \texttt{weka.classifiers.meta}), is similar to the
\texttt{FilteredClassifier}. But instead of taking a base-classifier and a
filter as parameters to perform the filtering, the
\texttt{AttributeSelectedClassifier} uses a \textit{search} algorithm (derived
from \texttt{weka.attributeSelection.ASEvaluation}), an \textit{evaluator}
(superclass is \texttt{weka.attributeSelection.ASSearch}) to perform the
attribute selection and a base-classifier to train on the reduced data.

This example here uses \texttt{J48} as base-classifier, \texttt{CfsSubsetEval}
as evaluator and a backwards operating \texttt{GreedyStepwise} as search method:
\begin{verbatim}
   import weka.attributeSelection.CfsSubsetEval;
   import weka.attributeSelection.GreedyStepwise;
   import weka.classifiers.Evaluation;
   import weka.classifiers.meta.AttributeSelectedClassifier;
   import weka.classifiers.trees.J48;
   import weka.core.Instances;
   ...
   Instances data = ... // from somewhere
   // setup meta-classifier
   AttributeSelectedClassifier classifier = new AttributeSelectedClassifier();
   CfsSubsetEval eval = new CfsSubsetEval();
   GreedyStepwise search = new GreedyStepwise();
   search.setSearchBackwards(true);
   J48 base = new J48();
   classifier.setClassifier(base);
   classifier.setEvaluator(eval);
   classifier.setSearch(search);
   // cross-validate classifier
   Evaluation evaluation = new Evaluation(data);
   evaluation.crossValidateModel(classifier, data, 10, new Random(1));
   System.out.println(evaluation.toSummaryString());
\end{verbatim}

\newpage

\subsection{Using the filter}
In case the data only needs to be reduced in dimensionality, but not used for
training a classifier, then the filter approach is the right one. The
\texttt{AttributeSelection} filter (package
\texttt{weka.filters.supervised.attribute}) takes an evaluator and a search
algorithm as parameter.

The code snippet below uses once again \texttt{CfsSubsetEval} as evaluator and
a backwards operating \texttt{GreedyStepwise} as search algorithm. It just
outputs the reduced data to \texttt{stdout} after the filtering step:
\begin{verbatim}
   import weka.attributeSelection.CfsSubsetEval;
   import weka.attributeSelection.GreedyStepwise;
   import weka.core.Instances;
   import weka.filters.Filter;
   import weka.filters.supervised.attribute.AttributeSelection;
   ...
   Instances data = ... // from somewhere
   // setup filter
   AttributeSelection filter = new AttributeSelection();
   CfsSubsetEval eval = new CfsSubsetEval();
   GreedyStepwise search = new GreedyStepwise();
   search.setSearchBackwards(true);
   filter.setEvaluator(eval);
   filter.setSearch(search);
   filter.setInputFormat(data);
   // filter data
   Instances newData = Filter.useFilter(data, filter);
   System.out.println(newData);
\end{verbatim}

\newpage

\subsection{Using the API directly}
Using the meta-classifier or the filter approach makes attribute selection
fairly easy. But it might not satify everybody's needs. For instance, if one
wants to obtain the ordering of the attributes (using \texttt{Ranker}) or
retrieve the indices of the selected attributes instead of the reduced data.

Just like the other examples, the one shown here uses the
\texttt{CfsSubsetEval} evaluator and the \texttt{GreedyStepwise} search
algorithm (in backwards mode). But instead of outputting the reduced data, only
the selected indices are printed in the console:
\begin{verbatim}
   import weka.attributeSelection.AttributeSelection;
   import weka.attributeSelection.CfsSubsetEval;
   import weka.attributeSelection.GreedyStepwise;
   import weka.core.Instances;
   ...
   Instances data = ... // from somewhere
   // setup attribute selection
   AttributeSelection attsel = new AttributeSelection();
   CfsSubsetEval eval = new CfsSubsetEval();
   GreedyStepwise search = new GreedyStepwise();
   search.setSearchBackwards(true);
   attsel.setEvaluator(eval);
   attsel.setSearch(search);
   // perform attribute selection
   attsel.SelectAttributes(data);
   int[] indices = attsel.selectedAttributes();
   System.out.println(
        "selected attribute indices (starting with 0):\n"
        + Utils.arrayToString(indices));
\end{verbatim}

\newpage

%%%%%%%%%%%%%%%
% Saving data %
%%%%%%%%%%%%%%%
\section{Saving data}
\label{api_saving_data}
Saving \texttt{weka.core.Instances} objects is as easy as reading the data in
the first place, though the process of storing the data again is far less common
than of reading the data into memory. The following two sections cover how to
save the data in files and in databases.

Just like with loading the data in chapter \ref{api_loading_data}, examples
classes for saving data can be found in the
\texttt{wekaexamples.core.converters} package of the \textit{Weka Examples}
collection\cite{wekaexamples};

\subsection{Saving data to files}
Once again, one can either let WEKA choose the appropriate converter for saving
the data or use an explicit converter (all savers are located in the
\texttt{weka.core.converters} package). The latter approach is necessary, if the
file name under which the data will be stored does not have an extension that
WEKA recognizes.

Use the \texttt{DataSink} class (inner class of
\texttt{weka.core.converters.ConverterUtils}), if the extensions are not a
problem. Here are a few examples:
\begin{verbatim}
   import weka.core.Instances;
   import weka.core.converters.ConverterUtils.DataSink;
   ...
   // data structure to save
   Instances data = ...
   // save as ARFF
   DataSink.write("/some/where/data.arff", data);
   // save as CSV
   DataSink.write("/some/where/data.csv", data);
\end{verbatim}
And here is an example of using the \texttt{CSVSaver} converter explicitly:
\begin{verbatim}
   import weka.core.Instances;
   import weka.core.converters.CSVSaver;
   import java.io.File;
   ...
   // data structure to save
   Instances data = ...
   // save as CSV
   CSVSaver saver = new CSVSaver();
   saver.setInstances(data);
   saver.setFile(new File("/some/where/data.csv"));
   saver.writeBatch();
\end{verbatim}

\subsection{Saving data to databases}
Apart from the KnowledgeFlow, saving to databases is not very obvious in WEKA,
unless one knows about the \texttt{DatabaseSaver} converter. Just like the
\texttt{DatabaseLoader}, the saver counterpart can store the data either in
batch mode or incrementally as well.

The first example shows how to save the data in batch mode, which is the easier
way of doing it:
\begin{verbatim}
   import weka.core.Instances;
   import weka.core.converters.DatabaseSaver;
   ...
   // data structure to save
   Instances data = ...
   // store data in database
   DatabaseSaver saver = new DatabaseSaver();
   saver.setDestination("jdbc_url", "the_user", "the_password");
   // we explicitly specify the table name here:
   saver.setTableName("whatsoever2");
   saver.setRelationForTableName(false);
   // or we could just update the name of the dataset:
   // saver.setRelationForTableName(true);
   // data.setRelationName("whatsoever2");
   saver.setInstances(data);
   saver.writeBatch();
\end{verbatim}
Saving the data incrementally, requires a bit more work, as one has to specify
that writing the data is done incrementally (using the \texttt{setRetrieval}
method), as well as notifying the saver when all the data has been saved:
\begin{verbatim}
   import weka.core.Instances;
   import weka.core.converters.DatabaseSaver;
   ...
   // data structure to save
   Instances data = ...
   // store data in database
   DatabaseSaver saver = new DatabaseSaver();
   saver.setDestination("jdbc_url", "the_user", "the_password");
   // we explicitly specify the table name here:
   saver.setTableName("whatsoever2");
   saver.setRelationForTableName(false);
   // or we could just update the name of the dataset:
   // saver.setRelationForTableName(true);
   // data.setRelationName("whatsoever2");
   saver.setRetrieval(DatabaseSaver.INCREMENTAL);
   saver.setStructure(data);
   count = 0;
   for (int i = 0; i < data.numInstances(); i++) {
     saver.writeIncremental(data.instance(i));
   }
   // notify saver that we're finished
   saver.writeIncremental(null);
\end{verbatim}

\newpage

%%%%%%%%%%%%%%%%%
% Visualization %
%%%%%%%%%%%%%%%%%
\section{Visualization}
\label{api_visualization}
The concepts covered in this chapter are also available through the example
classes of the \textit{Weka Examples} collection\cite{wekaexamples}. See the
following packages:
\begin{tight_itemize}
	\item \texttt{wekaexamples.gui.graphvisualizer}
	\item \texttt{wekaexamples.gui.treevisualizer}
	\item \texttt{wekaexamples.gui.visualize}
\end{tight_itemize}

\subsection{ROC curves}
WEKA can generate ``Receiver operating characteristic'' (ROC) curves, based on
the collected predictions during an evaluation of a classifier. In order to
display a ROC curve, one needs to perform the following steps:
\begin{tight_enumerate}
	\item Generate the plotable data based on the \texttt{Evaluation}'s
collected predictions, using the \texttt{ThresholdCurve} class (package
\texttt{weka.classifiers.evaluation}).
	\item Put the plotable data into a plot container, an instance of the
\texttt{PlotData2D} class (package \texttt{weka.gui.visualize}).
	\item Add the plot container to a visualization panel for displaying the
data, an instance of the \texttt{ThresholdVisualizePanel} class (package
\texttt{weka.gui.visualize}).
	\item Add the visualization panel to a \texttt{JFrame} (package
\texttt{javax.swing}) and display it.
\end{tight_enumerate}
And now, the four steps translated into actual code:
\begin{tight_enumerate}
	\item Generate the plotable data
	\begin{verbatim}
		Evaluation eval = ... // from somewhere
		ThresholdCurve tc = new ThresholdCurve();
		int classIndex = 0;  // ROC for the 1st class label
		Instances curve = tc.getCurve(eval.predictions(), classIndex);
	\end{verbatim}

	\item Put the plotable into a plot container
	\begin{verbatim}
		PlotData2D plotdata = new PlotData2D(curve);
		plotdata.setPlotName(curve.relationName());
		plotdata.addInstanceNumberAttribute();
	\end{verbatim}

	\item Add the plot container to a visualization panel
	\begin{verbatim}
		ThresholdVisualizePanel tvp = new ThresholdVisualizePanel();
		tvp.setROCString("(Area under ROC = " +
		  Utils.doubleToString(ThresholdCurve.getROCArea(curve),4)+")");
		tvp.setName(curve.relationName());
		tvp.addPlot(plotdata);
	\end{verbatim}

	\item Add the visualization panel to a \texttt{JFrame}
	\begin{verbatim}
		final JFrame jf = new JFrame("WEKA ROC: " + tvp.getName());
		jf.setSize(500,400);
		jf.getContentPane().setLayout(new BorderLayout());
		jf.getContentPane().add(tvp, BorderLayout.CENTER);
		jf.setDefaultCloseOperation(JFrame.DISPOSE_ON_CLOSE);
		jf.setVisible(true);
	\end{verbatim}
\end{tight_enumerate}

\newpage

\subsection{Graphs}
Classes implementing the \texttt{weka.core.Drawable} interface can generate
graphs of their internal models which can be displayed. There are two
different types of graphs available at the moment, which are explained in the
subsequent sections:
\begin{tight_itemize}
	\item Tree -- decision trees.
	\item BayesNet -- bayesian net graph structures.
\end{tight_itemize}

\subsubsection{Tree}
It is quite easy to display the internal tree structure of classifiers like
J48 or M5P (package \texttt{weka.classifiers.trees}). The following example
builds a J48 classifier on a dataset and displays the generated tree visually
using the \texttt{TreeVisualizer} class (package \texttt{weka.gui.treevisualizer}).
This visualization class can be used to view trees (or digraphs) in GraphViz's DOT
language\cite{dotformat}.
\begin{verbatim}
   import weka.classifiers.trees.J48;
   import weka.core.Instances;
   import weka.gui.treevisualizer.PlaceNode2;
   import weka.gui.treevisualizer.TreeVisualizer;
   import java.awt.BorderLayout;
   import javax.swing.JFrame;
   ...
   Instances data = ... // from somewhere
   // train classifier
   J48 cls = new J48();
   cls.buildClassifier(data);
   // display tree
   TreeVisualizer tv = new TreeVisualizer(
       null, cls.graph(), new PlaceNode2());
   JFrame jf = new JFrame("Weka Classifier Tree Visualizer: J48");
   jf.setDefaultCloseOperation(JFrame.DISPOSE_ON_CLOSE);
   jf.setSize(800, 600);
   jf.getContentPane().setLayout(new BorderLayout());
   jf.getContentPane().add(tv, BorderLayout.CENTER);
   jf.setVisible(true);
   // adjust tree
   tv.fitToScreen();
\end{verbatim}

\newpage

\subsubsection{BayesNet}
The graphs that the BayesNet classifier (package \texttt{weka.classifiers.bayes})
generates can be displayed using the \texttt{GraphVisualizer} class
(located in package \texttt{weka.gui.graphvisualizer}). The \texttt{GraphVisualizer} can
display graphs that are either in GraphViz's DOT language\cite{dotformat} or in
XML BIF\cite{cozman} format. For displaying DOT format, one needs to use the
method \texttt{readDOT}, and for the BIF format the method \texttt{readBIF}.

The following code snippet trains a \texttt{BayesNet} classifier on some data
and then displays the graph generated from this data in a frame:
\begin{verbatim}
   import weka.classifiers.bayes.BayesNet;
   import weka.core.Instances;
   import weka.gui.graphvisualizer.GraphVisualizer;
   import java.awt.BorderLayout;
   import javax.swing.JFrame;
   ...
   Instances data = ... // from somewhere
   // train classifier
   BayesNet cls = new BayesNet();
   cls.buildClassifier(data);
   // display graph
   GraphVisualizer gv = new GraphVisualizer();
   gv.readBIF(cls.graph());
   JFrame jf = new JFrame("BayesNet graph");
   jf.setDefaultCloseOperation(JFrame.DISPOSE_ON_CLOSE);
   jf.setSize(800, 600);
   jf.getContentPane().setLayout(new BorderLayout());
   jf.getContentPane().add(gv, BorderLayout.CENTER);
   jf.setVisible(true);
   // layout graph
   gv.layoutGraph();
\end{verbatim}

\newpage

%%%%%%%%%%%%%%%%%
% Serialization %
%%%%%%%%%%%%%%%%%
\section{Serialization}
\label{api_serialization}

\textit{Serialization}\footnote{\url{http://en.wikipedia.org/wiki/Serialization}
{}} is the process of saving an object in a persistent form, e.g., on the
harddisk as a bytestream. \textit{Deserialization} is the process in the
opposite direction, creating an object from a persistently saved data structure.
In Java, an object can be serialized if it imports the
\texttt{java.io.Serializable} interface. Members of an object that are not
supposed to be serialized, need to be declared with the keyword
\texttt{transient}.

In the following are some Java code snippets for serializing and
deserializing a \texttt{J48} classifier. Of course, serialization is not limited
to classifiers. Most schemes in WEKA, like clusterers and filters, are also
serializable.

\subsubsection*{Serializing a classifier}
The \texttt{weka.core.SerializationHelper} class makes it easy to serialize an
object. For saving, one can use one of the \texttt{write} methods:
\begin{verbatim}
   import weka.classifiers.Classifier;
   import weka.classifiers.trees.J48;
   import weka.core.converters.ConverterUtils.DataSource;
   import weka.core.SerializationHelper;
   ...
   // load data
   Instances inst = DataSource.read("/some/where/data.arff");
   inst.setClassIndex(inst.numAttributes() - 1);
   // train J48
   Classifier cls = new J48();
   cls.buildClassifier(inst);
   // serialize model
   SerializationHelper.write("/some/where/j48.model", cls);
\end{verbatim}

\subsubsection*{Deserializing a classifier}
Deserializing an object can be achieved by using one of the \texttt{read}
methods:
\begin{verbatim}
   import weka.classifiers.Classifier;
   import weka.core.SerializationHelper;
   ...
   // deserialize model
   Classifier cls = (Classifier) SerializationHelper.read(
       "/some/where/j48.model");
\end{verbatim}

\newpage

\subsubsection*{Deserializing a classifier saved from the Explorer}
The Explorer does not only save the built classifier in the model
file, but also the header information of the dataset the classifier was built
with. By storing the dataset information as well, one can easily check whether a
serialized classifier can be applied on the current dataset. The \texttt{readAll}
methods returns an array with all objects that are contained in the model file.
\begin{verbatim}
   import weka.classifiers.Classifier;
   import weka.core.Instances;
   import weka.core.SerializationHelper;
   ...
   // the current data to use with classifier
   Instances current = ... // from somewhere
   // deserialize model
   Object o[] = SerializationHelper.readAll("/some/where/j48.model");
   Classifier cls = (Classifier) o[0];
   Instances data = (Instances) o[1];
   // is the data compatible?
   if (!data.equalHeaders(current))
      throw new Exception("Incompatible data!");
\end{verbatim}

\subsubsection*{Serializing a classifier for the Explorer}
If one wants to serialize the dataset header information alongside the
classifier, just like the Explorer does, then one can use one of the
\texttt{writeAll} methods:
\begin{verbatim}
   import weka.classifiers.Classifier;
   import weka.classifiers.trees.J48;
   import weka.core.converters.ConverterUtils.DataSource;
   import weka.core.SerializationHelper;
   ...
   // load data
   Instances inst = DataSource.read("/some/where/data.arff");
   inst.setClassIndex(inst.numAttributes() - 1);
   // train J48
   Classifier cls = new J48();
   cls.buildClassifier(inst);
   // serialize classifier and header information
   Instances header = new Instances(inst, 0);
   SerializationHelper.writeAll(
       "/some/where/j48.model", new Object[]{cls, header});
\end{verbatim}
