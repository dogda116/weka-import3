%
%    This program is free software; you can redistribute it and/or modify
%    it under the terms of the GNU General Public License as published by
%    the Free Software Foundation; either version 2 of the License, or
%    (at your option) any later version.
%
%    This program is distributed in the hope that it will be useful,
%    but WITHOUT ANY WARRANTY; without even the implied warranty of
%    MERCHANTABILITY or FITNESS FOR A PARTICULAR PURPOSE.  See the
%    GNU General Public License for more details.
%
%    You should have received a copy of the GNU General Public License
%    along with this program; if not, write to the Free Software
%    Foundation, Inc., 675 Mass Ave, Cambridge, MA 02139, USA.
%

% Version: $Revision$

The ``work horses'' of preprocessing in WEKA are \textit{filters}. They perform
many tasks, from resampling data, to deleting and standardizing attributes. In
the following are two different approaches covered that explain in detail how
to implement a new filter:
\begin{tight_itemize}
  \item \textbf{default} -- this is how filters had to be implemented in the
past.
  \item \textbf{simple }-- since there are mainly two types of filters, batch or
stream, additional abstract classes were introduced to speed up
the implementation process.
\end{tight_itemize}

\subsection{Default approach}
The \textit{default} approach is the most flexible, but also the most 
complicated one for writing a new filter. This approach has to be used, if the
filter cannot be written using the \textit{simple} approach described further
below.

\subsubsection{Implementation}
The following methods are of importance for the implementation of a filter and
explained in detail further down. It is also a good idea studying the Javadoc
of these methods as declared in the \texttt{weka.filters.Filter} class:
\begin{tight_itemize}
  \item \texttt{getCapabilities()}
  \item \texttt{setInputFormat(Instances)}
  \item \texttt{getInputFormat()}
  \item \texttt{setOutputFormat(Instances)}
  \item \texttt{getOutputFormat()}
  \item \texttt{input(Instance)}
  \item \texttt{bufferInput(Instance)}
  \item \texttt{push(Instance)}
  \item \texttt{output()}
  \item \texttt{batchFinished()}
  \item \texttt{flushInput()}
  \item \texttt{getRevision()}
\end{tight_itemize}
But only the following ones normally need to be modified:
\begin{tight_itemize}
  \item \texttt{getCapabilities()}
  \item \texttt{setInputFormat(Instances)}
  \item \texttt{input(Instance)}
  \item \texttt{batchFinished()}
  \item \texttt{getRevision()}
\end{tight_itemize}
For more information on ``Capabilities'' see section \ref{filter_capabilities}.
Please note, that the \texttt{weka.filters.Filter} superclass does not
implement the \texttt{weka.core.OptionHandler} interface. See section ``Option
handling'' on page \pageref{filter_optionhandling}.

\newpage
\heading{setInputFormat(Instances)}
With this call, the user tells the filter what structure, i.e., attributes, the
input data has. This method also tests, whether the filter can actually process
this data, according to the capabilities specified in the
\texttt{getCapabilities()} method.

If the output format of the filter, i.e., the new \texttt{Instances} header, can
be determined based alone on this information, then the method should set the
output format via \texttt{setOutputFormat(Instances)} and return \texttt{true},
otherwise it has to return \texttt{false}.

\heading{getInputFormat()}
This method returns an \texttt{Instances} object containing all currently
buffered \texttt{Instance} objects from the input queue.

\heading{setOutputFormat(Instances)}
\texttt{setOutputFormat(Instances)} defines the new \texttt{Instances} header
for the output data. For filters that work on a row-basis, there should not be
any changes between the input and output format. But filters that work on
attributes, e.g., removing, adding, modifying, will affect this format. This
method must be called with the appropriate \texttt{Instances} object as
parameter, since all \texttt{Instance} objects being processed will rely on the
output format (they use it as dataset that they belong to).

\heading{getOutputFormat()}
This method returns the currently set \texttt{Instances} object that defines the
output format. In case \texttt{setOutputFormat(Instances)} has not been called
yet, this method will return \texttt{null}.

\heading{input(Instance)} returns \texttt{true} if the given
\texttt{Instance} can be processed straight away and can be collected
immediately via the \texttt{output()} method (after adding it to the output
queue via \texttt{push(Instance)}, of course). This is also the case if the
first batch of data has been processed and the \texttt{Instance }belongs to the
second batch. Via \texttt{isFirstBatchDone()} one can query whether this
\texttt{Instance} is still part of the first batch or of the second.

If the \texttt{Instance} cannot be processed immediately, e.g., the filter needs
to collect all the data first before doing some calculations, then it needs to
be buffered with \texttt{bufferInput(Instance)} until \texttt{batchFinished()}
is called. In this case, the method needs to return \texttt{false}.

\heading{bufferInput(Instance)}
In case an \texttt{Instance} cannot be processed immediately, one can use this
method to buffer them in the input queue. All buffered \texttt{Instance} objects
are available via the \texttt{getInputFormat()} method.

\heading{push(Instance)}
adds the given \texttt{Instance} to the output queue.

\heading{output()}
Returns the next \texttt{Instance} object from the output queue and removes it
from there. In case there is no \texttt{Instance} available this method returns
\texttt{null}.

\newpage
\heading{batchFinished()} signals the end of a dataset being pushed
through the filter. In case of a filter that could not process the data of the
first batch immediately, this is the place to determine what the output format
will be (and set if via \texttt{setOutputFormat(Instances)}) and finally process
the input data. The currently available data can be retrieved with the
\texttt{getInputFormat()} method. After processing the data, one needs to call
\texttt{flushInput()} to remove all the pending input data.

\heading{flushInput()}
\texttt{flushInput()} removes all buffered \texttt{Instance} objects from the
input queue. This method must be called after all the \texttt{Instance} objects
have been processed in the \texttt{batchFinished()} method.

\subsubsection*{Option handling}
\label{filter_optionhandling}
If the filter should be able to handle command-line options, then the interface
\texttt{weka.core.OptionHandler} needs to be implemented. In addition
to that, the following code should be added at the end of the
\texttt{setOptions(String[])} method:
\begin{verbatim}
if (getInputFormat() != null) {
    setInputFormat(getInputFormat());
}
\end{verbatim}
This will inform the filter about changes in the options and therefore reset it.

\newpage
\subsubsection{Examples}
The following examples, covering batch and stream filters, illustrate the
filter framework and how to use it.

Unseeded random number generators like \texttt{Math.random()} should
\textbf{never} be used since they will produce different results in each run and
repeatable experiments are essential in machine learning.

\subsubsection*{BatchFilter}
This simple batch filter adds a new attribute called \textit{blah} at the end of
the dataset. The rows of this attribute contain only the row's index in the
data. Since the batch-filter does not have to see all the data before creating
the output format, the \texttt{setInputFormat(Instances)} sets the output format
and returns \texttt{true} (indicating that the output format can be queried
immediately). The \texttt{batchFinished()} method performs the processing of all
the data.

{\scriptsize % Version: $Revision$

\begin{verbatim}
import weka.core.*;
import weka.core.Capabilities.*;

public class BatchFilter extends Filter {

  public String globalInfo() {
    return   "A batch filter that adds an additional attribute 'blah' at the end "
           + "containing the index of the processed instance. The output format "
           + "can be collected immediately.";
  }

  public Capabilities getCapabilities() {
    Capabilities result = super.getCapabilities();
    result.enableAllAttributes();
    result.enableAllClasses();
    result.enable(Capability.NO_CLASS);  // filter doesn't need class to be set
    return result;
  }

  public boolean setInputFormat(Instances instanceInfo) throws Exception {
    super.setInputFormat(instanceInfo);
    Instances outFormat = new Instances(instanceInfo, 0);
    outFormat.insertAttributeAt(new Attribute("blah"),
	  outFormat.numAttributes());
    setOutputFormat(outFormat);
    return true;  // output format is immediately available
  }

  public boolean batchFinished() throws Exception {
    if (getInputFormat() = null)
      throw new NullPointerException("No input instance format defined");
    Instances inst = getInputFormat();
    Instances outFormat = getOutputFormat();
    for (int i = 0; i < inst.numInstances(); i++) {
      double[] newValues = new double[outFormat.numAttributes()];
      double[] oldValues = inst.instance(i).toDoubleArray();
      System.arraycopy(oldValues, 0, newValues, 0, oldValues.length);
      newValues[newValues.length - 1] = i;
      push(new Instance(1.0, newValues));
    }
    flushInput();
    m_NewBatch = true;
    m_FirstBatchDone = true;
    return (numPendingOutput() != 0);
  }

  public static void main(String[] args) {
    runFilter(new BatchFilter(), args);
  }
}
\end{verbatim}
}

\newpage
\subsubsection*{BatchFilter2}
In contrast to the first batch filter, this one here cannot determine the output
format immediately (the number of instances in the first batch is part of the
attribute name now). This is done in the \texttt{batchFinished()} method.

{\scriptsize % Version: $Revision$

\begin{verbatim}
import weka.core.*;
import weka.core.Capabilities.*;

public class BatchFilter2 extends Filter {

  public String globalInfo() {
    return   "A batch filter that adds an additional attribute 'bla' at the end "
           + "containing the index of the processed instance. The output format "
           + "cannot be collected immediately.";
  }

  public Capabilities getCapabilities() {
    Capabilities result = super.getCapabilities();
    result.enableAllAttributes();
    result.enableAllClasses();
    result.enable(Capability.NO_CLASS);  // filter doesn't need class to be set
    return result;
  }

  public boolean batchFinished() throws Exception {
    if (getInputFormat() = null)
      throw new NullPointerException("No input instance format defined");
    // output format still needs to be set (depends on first batch of data)
    if (!isFirstBatchDone()) {
      Instances outFormat = new Instances(getInputFormat(), 0);
      outFormat.insertAttributeAt(new Attribute(
          "bla-" + getInputFormat().numInstances()), outFormat.numAttributes());
      setOutputFormat(outFormat);
    }
    Instances inst = getInputFormat();
    Instances outFormat = getOutputFormat();
    for (int i = 0; i < inst.numInstances(); i++) {
      double[] newValues = new double[outFormat.numAttributes()];
      double[] oldValues = inst.instance(i).toDoubleArray();
      System.arraycopy(oldValues, 0, newValues, 0, oldValues.length);
      newValues[newValues.length - 1] = i;
      push(new Instance(1.0, newValues));
    }
    flushInput();
    m_NewBatch = true;
    m_FirstBatchDone = true;
    return (numPendingOutput() != 0);
  }

  public static void main(String[] args) {
    runFilter(new BatchFilter2(), args);
  }
}
\end{verbatim}
}

\newpage
\subsubsection*{BatchFilter3}
As soon as this batch filter's first batch is done, it can process
\texttt{Instance} objects immediately in the \texttt{input(Instance)} method. It
adds a new attribute which contains just a random number, but the random number
generator being used is seeded with the number of instances from the first
batch.

{\scriptsize % Version: $Revision$

\begin{verbatim}
import weka.core.*;
import weka.core.Capabilities.*;
import java.util.Random;

public class BatchFilter3 extends Filter {

  protected int m_Seed;
  protected Random m_Random;

  public String globalInfo() {
    return   "A batch filter that adds an attribute 'bla' at the end "
           + "containing a random number. The output format cannot be collected "
           + "immediately.";
  }

  public Capabilities getCapabilities() {
    Capabilities result = super.getCapabilities();
    result.enableAllAttributes();
    result.enableAllClasses();
    result.enable(Capability.NO_CLASS);  // filter doesn't need class to be set
    return result;
  }

  public boolean input(Instance instance) throws Exception {
    if (getInputFormat() = null)
      throw new NullPointerException("No input instance format defined");
    if (isNewBatch()) {
      resetQueue();
      m_NewBatch = false;
    }
    if (isFirstBatchDone())
      convertInstance(instance);
    else
      bufferInput(instance);
    return isFirstBatchDone();
  }

  public boolean batchFinished() throws Exception {
    if (getInputFormat() = null)
      throw new NullPointerException("No input instance format defined");
    // output format still needs to be set (random number generator is seeded
    // with number of instances of first batch)
    if (!isFirstBatchDone()) {
      m_Seed = getInputFormat().numInstances();
      Instances outFormat = new Instances(getInputFormat(), 0);
      outFormat.insertAttributeAt(new Attribute(
          "bla-" + getInputFormat().numInstances()), outFormat.numAttributes());
      setOutputFormat(outFormat);
    }
    Instances inst = getInputFormat();
    for (int i = 0; i < inst.numInstances(); i++) {
      convertInstance(inst.instance(i));
    }
    flushInput();
    m_NewBatch = true;
    m_FirstBatchDone = true;
    m_Random = null;
    return (numPendingOutput() != 0);
  }

  protected void convertInstance(Instance instance) {
    if (m_Random = null)
      m_Random = new Random(m_Seed);
    double[] newValues = new double[instance.numAttributes() + 1];
    double[] oldValues = instance.toDoubleArray();
    newValues[newValues.length - 1] = m_Random.nextInt();
    System.arraycopy(oldValues, 0, newValues, 0, oldValues.length);
    push(new Instance(1.0, newValues));
  }

  public static void main(String[] args) {
    runFilter(new BatchFilter3(), args);
  }
}
\end{verbatim}
}

\newpage
\subsubsection*{StreamFilter}
This stream filter adds a random number (the seed value is hard-coded) at the
end of each \texttt{Instance} of the input data. Since this does not rely on
having access to the full data of the first batch, the output format is
accessible immediately after using \texttt{setInputFormat(Instances)}. All the
\texttt{Instance} objects are immediately processed in \texttt{input(Instance)}
via the \texttt{convertInstance(Instance)} method, which pushes them immediately
to the output queue.

{\scriptsize % Version: $Revision$

\begin{verbatim}
import weka.core.*;
import weka.core.Capabilities.*;
import java.util.Random;

public class StreamFilter extends Filter {

  protected Random m_Random;

  public String globalInfo() {
    return   "A stream filter that adds an attribute 'blah' at the end "
           + "containing a random number. The output format can be collected "
           + "immediately.";
  }

  public Capabilities getCapabilities() {
    Capabilities result = super.getCapabilities();
    result.enableAllAttributes();
    result.enableAllClasses();
    result.enable(Capability.NO_CLASS);  // filter doesn't need class to be set
    return result;
  }

  public boolean setInputFormat(Instances instanceInfo) throws Exception {
    super.setInputFormat(instanceInfo);
    Instances outFormat = new Instances(instanceInfo, 0);
    outFormat.insertAttributeAt(new Attribute("blah"),
	  outFormat.numAttributes());
    setOutputFormat(outFormat);
    m_Random = new Random(1);
    return true;  // output format is immediately available
  }

  public boolean input(Instance instance) throws Exception {
    if (getInputFormat() = null)
      throw new NullPointerException("No input instance format defined");
    if (isNewBatch()) {
      resetQueue();
      m_NewBatch = false;
    }
    convertInstance(instance);
    return true;  // can be immediately collected via output()
  }

  protected void convertInstance(Instance instance) {
    double[] newValues = new double[instance.numAttributes() + 1];
    double[] oldValues = instance.toDoubleArray();
    newValues[newValues.length - 1] = m_Random.nextInt();
    System.arraycopy(oldValues, 0, newValues, 0, oldValues.length);
    push(new Instance(1.0, newValues));
  }

  public static void main(String[] args) {
    runFilter(new StreamFilter(), args);
  }
}
\end{verbatim}
}

\newpage
\subsection{Simple approach}
The base filters and interfaces are all located in the following package:
\begin{verbatim}
 weka.filters
\end{verbatim}
One can basically divide filters roughly into two different kinds of
filters:
\begin{tight_itemize}
  \item \textbf{batch filters} -- they need to see the whole dataset before they
can start processing it, which they do in one go
  \item \textbf{stream filters} -- they can start producing output right away
and the data just passes through while being modified
\end{tight_itemize}
You can subclass one of the following abstract filters, depending on the kind of
classifier you want to implement:
\begin{tight_itemize}
  \item \texttt{weka.filters.SimpleBatchFilter}
  \item \texttt{weka.filters.SimpleStreamFilter}
\end{tight_itemize}
These filters simplify the rather general and complex framework introduced by
the abstract superclass \texttt{weka.filters.Filter}. One only needs to
implement a couple of abstract methods that will process the actual data and
override, if necessary, a few existing methods for option handling.

\subsubsection{SimpleBatchFilter}
Only the following abstract methods need to be implemented:
\begin{tight_itemize}
  \item \texttt{globalInfo()} -- returns a short description of what the
filter does; will be displayed in the GUI
  \item \texttt{determineOutputFormat(Instances)} -- generates the new
format, based on the input data
  \item \texttt{process(Instances)} -- processes the whole dataset in one
go
  \item \texttt{getRevision()} -- returns the Subversion revision information,
see section ``Revisions'' on page \pageref{filter_revisions}
\end{tight_itemize}
If more options are necessary, then the following methods need to be overridden:
\begin{tight_itemize}
  \item \texttt{listOptions()} -- returns an enumeration of the available
options; these are printed if one calls the filter with the -h option
  \item \texttt{setOptions(String[])} -- parses the given option array,
that were passed from command-line
  \item \texttt{getOptions()} -- returns an array of options, resembling
the current setup of the filter
\end{tight_itemize}
See section ``Methods'' on page \pageref{classifier_methods} and section
``Parameters'' on page \pageref{classifier_parameters} for more information.

\newpage
In the following an example implementation that adds an additional attribute at
the end, containing the index of the processed instance:

{\footnotesize % Version: $Revision$

\begin{verbatim}
import weka.core.*;
import weka.core.Capabilities.*;
import weka.filters.*;

public class SimpleBatch extends SimpleBatchFilter {

  public String globalInfo() {
    return   "A simple batch filter that adds an additional attribute 'blah' at the end "
           + "containing the index of the processed instance.";
  }

  public Capabilities getCapabilities() {
    Capabilities result = super.getCapabilities();
    result.enableAllAttributes();
    result.enableAllClasses();
    result.enable(Capability.NO_CLASS);  //// filter doesn't need class to be set//
    return result;
  }

  protected Instances determineOutputFormat(Instances inputFormat) {
    Instances result = new Instances(inputFormat, 0);
    result.insertAttributeAt(new Attribute("blah"), result.numAttributes());
    return result;
  }

  protected Instances process(Instances inst) {
    Instances result = new Instances(determineOutputFormat(inst), 0);
    for (int i = 0; i < inst.numInstances(); i++) {
      double[] values = new double[result.numAttributes()];
      for (int n = 0; n < inst.numAttributes(); n++)
        values[n] = inst.instance(i).value(n);
      values[values.length - 1] = i;
      result.add(new Instance(1, values));
    }
    return result;
  }

  public static void main(String[] args) {
    runFilter(new SimpleBatch(), args);
  }
}
\end{verbatim}
}

\newpage
\subsubsection{SimpleStreamFilter}
Only the following abstract methods need to be implemented for a stream filter:
\begin{tight_itemize}
  \item \texttt{globalInfo()} -- returns a short description of what the filter
does; will be displayed in the GUI
  \item \texttt{determineOutputFormat(Instances)} -- generates the new
format, based on the input data
  \item \texttt{process(Instance)} -- processes a single instance and turns it
from the old format into the new one
  \item \texttt{getRevision()} -- returns the Subversion revision
information, see section ``Revisions'' on page \pageref{filter_revisions}
\end{tight_itemize}
If more options are necessary, then the following methods need to be overridden:
\begin{tight_itemize}
  \item \texttt{listOptions()} -- returns an enumeration of the available
options; these are printed if one calls the filter with the -h option
  \item \texttt{setOptions(String[])} -- parses the given option array,
that were passed from command-line
  \item \texttt{getOptions()} -- returns an array of options, resembling the
current setup of the filter
\end{tight_itemize}
See also section \ref{classifier_methods}, covering ``Methods'' for classifiers.

\newpage
In the following an example implementation of a stream filter that adds an extra
attribute at the end, which is filled with random numbers. The \texttt{reset()}
method is only used in this example, since the random number generator needs to
be re-initialized in order to obtain repeatable results.

{\footnotesize % Version: $Revision$

\begin{verbatim}
import weka.core.*;
import weka.core.Capabilities.*;
import weka.filters.*;
import java.util.Random;

public class SimpleStream extends SimpleStreamFilter {

  protected Random m_Random;

  public String globalInfo() {
    return   "A simple stream filter that adds an attribute 'bla' at the end "
           + "containing a random number.";
  }

  public Capabilities getCapabilities() {
    Capabilities result = super.getCapabilities();
    result.enableAllAttributes();
    result.enableAllClasses();
    result.enable(Capability.NO_CLASS);  //// filter doesn't need class to be set//
    return result;
  }

  protected void reset() {
    super.reset();
    m_Random = new Random(1);
  }

  protected Instances determineOutputFormat(Instances inputFormat) {
    Instances result = new Instances(inputFormat, 0);
    result.insertAttributeAt(new Attribute("bla"), result.numAttributes());
    return result;
  }

  protected Instance process(Instance inst) {
    double[] values = new double[inst.numAttributes() + 1];
    for (int n = 0; n < inst.numAttributes(); n++)
      values[n] = inst.value(n);
    values[values.length - 1] = m_Random.nextInt();
    Instance result = new Instance(1, values);
    return result;
  }

  public static void main(String[] args) {
    runFilter(new SimpleStream(), args);
  }
}
\end{verbatim}
}

\noindent A real-world implementation of a stream filter is the
\texttt{MultiFilter} class (package \texttt{weka.filters}), which passes the
data through all the filters it contains. Depending on whether all the used
filters are streamable or not, it acts either as \textit{stream} filter or as
\textit{batch} filter.

\newpage
\subsubsection{Internals}
Some useful methods of the filter classes:
\begin{tight_itemize}
  \item \texttt{isNewBatch()} -- returns \texttt{true} if an instance of the
filter was just instantiated or a new batch was started via the
\texttt{batchFinished()} method.
  \item \texttt{isFirstBatchDone()} -- returns \texttt{true} as soon as the
first batch was finished via the \texttt{batchFinished()} method. Useful for
supervised filters, which should not be altered after being trained with the
first batch of instances.
\end{tight_itemize}

\subsection{Capabilities}
\label{filter_capabilities}
Filters implement the \texttt{weka.core.CapabilitiesHandler} interface like the
classifiers. This method returns what kind of data the filter is able to
process. Needs to be adapted for each individual filter, since the default
implementation allows the processing of all kinds of attributes and classes.
Otherwise correct functioning of the filter cannot be guaranteed. See section
``Capabilities'' on page \pageref{classifier_capabilities} for more information.

\subsection{Packages}
A few comments about the different filter sub-packages:
\begin{tight_itemize}
  \item \textbf{supervised} -- contains supervised filters, i.e., filters that
take class distributions into account. Must implement the
\texttt{weka.filters.SupervisedFilter} interface.
  \begin{tight_itemize}
	\item \textbf{attribute} -- filters that work column-wise.
	\item \textbf{instance} -- filters that work row-wise.
  \end{tight_itemize}
  \item \textbf{unsupervised} -- contains unsupervised filters, i.e., they work
without taking any class distributions into account. The filter must implement
the \texttt{weka.filters.UnsupervisedFilter} interface.
  \begin{tight_itemize}
	\item \textbf{attribute} -- filters that work column-wise.
	\item \textbf{instance} -- filters that work row-wise.
  \end{tight_itemize}
\end{tight_itemize}

\subsubsection*{Javadoc}
The Javadoc generation works the same as with classifiers. See section
``Javadoc'' on page \pageref{classifier_javadoc} for more information.

\subsection{Revisions}
\label{filter_revisions}
Filters, like classifiers, implement the \texttt{weka.core.RevisionHandler}
interface. This provides the functionality of obtaining the Subversion revision
from within Java. Filters that are not part of the official WEKA distribution
do not have to implement the method \texttt{getRevision()} as the
\texttt{weka.filters.Filter} class already implements this method.
Contributions, on the other hand, need to implement it, in order to
obtain the revision of this particular source file. See section ``Revisions''
on page \pageref{classifier_revisions}.

\newpage
\subsection{Testing}
WEKA provides already a test framework to ensure correct basic functionality of
a filter. It is essential for the filter to pass these tests.

\subsubsection{Option handling}
You can check the option handling of your filter with the following tool from
command-line:
\begin{verbatim}
 weka.core.CheckOptionHandler -W classname [-- additional parameters]
\end{verbatim}
All tests need to return \textit{yes}.

\subsubsection{GenericObjectEditor}
The \texttt{CheckGOE} class checks whether all the properties available in the
GUI have a tooltip accompanying them and whether the \texttt{globalInfo()}
method is declared:
\begin{verbatim}
 weka.core.CheckGOE -W classname [-- additional parameters]
\end{verbatim}
All tests, once again, need to return \textit{yes}.

\subsubsection{Source code}
Filters that implement the \texttt{weka.filters.Sourcable} interface can output
Java code of their internal representation. In order to check the generated
code, one should not only compile the code, but also test it with the following
test class:
\begin{verbatim}
 weka.filters.CheckSource
\end{verbatim}
This class takes the original WEKA filter, the generated code and the dataset
used for generating the source code (and an optional class index) as parameters.
It builds the WEKA filter on the dataset and compares the output, the one from
the WEKA filter and the one from the generated source code, whether they are the
same.

Here is an example call for
\texttt{weka.filters.unsupervised.attribute.ReplaceMissingValues} and the
generated class \texttt{weka.filters.WEKAWrapper} (it wraps the actual generated
code in a pseudo-filter):
\begin{verbatim}
 java weka.filters.CheckSource \
    -W weka.filters.unsupervised.attribute.ReplaceMissingValues \
    -S weka.filters.WEKAWrapper \
    -t data.arff
\end{verbatim}
It needs to return \textit{Tests OK!}.

\subsubsection{Unit tests}
In order to make sure that your filter applies to the WEKA criteria, you
should add your filter to the junit unit test framework, i.e., by creating a
Test class. The superclass for filter unit tests is
\texttt{weka.filters.AbstractFilterTest}.

